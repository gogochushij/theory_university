\lecture{3}{19 nov}{\dag}

\section{Теорема Кука-Левина}
$ \tThreeSAT = \{(F, A) \mid F \text{ --- в 3-КНФ}, F(A) = 1\}$.\index{$\tThreeSAT$}
\begin{ex}
    \[
		\left( (\neg x \vee \neg y \vee \neg z) \wedge (y \vee \neg z) \wedge (z), [x = 0, y= 1,  z=1] \right) \in  \tThreeSAT 
    .\] 
\end{ex}
\begin{thm}[Кук-Левин]\index{теорема Кука-Левина}
	$ \tThreeSAT $ --- $\tNP$-полная задача. 
\end{thm}
\begin{proof}
	Мы уже доказали полноту задач выполнимости булевых схем, поэтому будем сводить к ней.

	Пусть у нас есть некоторая схема. Для каждого гейта заведем по переменной, которая обозначает результат операции в этом гейте. Входы тоже остаются гейтами в схеме.

	Запишем для гейтов клозы длины 3, которые выражают результат в зависимости от аргументов.

	Например, для входов $ x$, $ y$ и операции $ \oplus = g(x, y)$, 
	\[
	\begin{aligned}
		&(x \vee y \vee \neg g) \\
		&(\neg x \vee \neg y \vee \neg g) \\
		&(x \vee \neg y\vee g) \\
		&(\neg x \vee y \vee g)
	\end{aligned}
	\]

	Еще для последнего гейта $ g$ (выходного) запишем клоз $ (g)$.

	Значение в полученной схеме будет соответствовать результату конъюнкции всех клозов и наоборот: по формуле можем построить булеву схему и входные переменные выполняют ее.
\end{proof}

\begin{thm}
	Пусть $ R \in \tNP$, и соответствующий язык $ L(R) $~--- \NP-полон. 
	Тогда  $ R$ сводится по Тьюрингу к $ L(R)$.
\end{thm}
\begin{proof}
	Во-первых, задача поиска из $ \NP$ сводится к $ \tSAT$. При этом $ \SAT$ сводится к $ L(R)$.
	
	Осталось свести $ \tSAT \to \SAT$. 
	Для этого нужно построить МТ с оракулом $\SAT$ будет решать $\tSAT$.
	
	Пусть нам дали формулу $ F$ и мы хотим найти выполняющий набор. Предполагается, что она выполнима.
	
	Подставим первую переменную $ x_1$  как $0$ и как $ 1$ в $ F$ (это тоже будут формулы) и спросим у оракула $ \SAT$ про $ F[x_1=0] \in \SAT$  и $ F[x_1=1] \in \SAT$.

	Так как $ F$ выполнима, хотя бы одна из полученных схем выполнима. Выберем ее и продолжим подставлять в нее. Так мы дойдем до конечной истиной формулы. Следовательно, последовательность подставляемых значений $ x_i$ и будет выполняющим набором.
\end{proof}

\subsection{Оптимальный алгоритм для $\tNP$-задачи}
Пусть мы хотим решить задачу, заданную отношениям $ R$, с входом $ x$.
Давайте переберем все машины (не только полиномиальные) и, если какая-то машина выдала результат $ y $, то проверим его $ R(x, y)$ за полином.

Если ответ подошел, то просто заканчиваем работу, иначе продолжаем ждать других результатов.

Если машины были бы запущены параллельно, то мы бы нашли ответ за время самой быстрой машины на данном входе.

А мы будем делать шаги <<змейкой>>: выделим для $ l$-ой машины $ 2^{-l}$ времени удельно.

На очередном этапе $ 2^{l}(1 + 2k) = 2^l + 2^{l+1}k$ будем моделировать $ k$-ый шаг машины $ M_l$.
То есть для $l$-той машины первый шаг (нулевой) будет сделан на $2^l$-том реальном шаге, и каждый следующий будет совершаться каждые $2^{l+1}$ шаг.

Так, например, будет выглядеть табличка для первых машин (указаны этапы моделирования, на которых будет выполнен соответствующий шаг):
\begin{center}\begin{tabular}{c|c|c|c|c}
     № МТ & 0 шаг & 1 шаг & 2 шаг & 3 шаг  \\
     \hline 
     0 & 1 & 3 & 5 & 7 \\
     1 & 2 & 6 & 10 & 14 \\
     2 & 4 & 12 & 20 & 28 \\
     3 & 8 & 24 & 40 & 56 
\end{tabular}\end{center}

Посчитаем замедление алгоритма. Мы хотим выдать ответ не сильно позже, чем самая быстрая машина.
Но заметим, что такая машина одна\footnote{Кажется, здесь важно просто, что такая есть и она имеет какой-то номер $l$.}, поэтому $ l$ это константа, следовательно, множитель $ 2^{l}$ тоже константа.

Как моделировать эти машины? Если было бы быстрое обращение к каждому элементу памяти, то $ t(x) \le \const_i \cdot t_i + p(\lvert x \rvert )$, где $ i$ - номер самой быстрой машины ($ p(\lvert x \rvert )$ на проверку). 

Но у нас ДМТ, поэтому получаем  $ t(x) \le \const_{i}\cdot p(t_i(x))$ из-за необходимости хранить на одной ленте несколько МТ и следовательно тратить время на <<переключения>> между ними.

\begin{note}
	Если $ \P = \NP$, то построенный алгоритм может решить $\SAT$ за полиномиальное от времени работы самой быстрой машины  $ p(t_i(x))$, но и оно полиномиально в случае $ \P = \NP$.
\end{note}

\begin{thm}
	Если $ \P \ne \NP$, то существует язык $ L \in \NP \setminus \P$, \textbf{не} являющийся $ \NP$-полным.
\end{thm}
\begin{proof}
	Найдем задачу, которая и не в $ \P$ и не является  $ \NP$-полной. 

	Занумеруем все полиномиальные ДМТ с полиномиальными будильниками\footnote{то есть настроенными на полиномиальное время}:
	\[
	M_1, M_2, \ldots 
	.\] 
	Аналогично пронумеруем полиномиальные сведения с будильниками, про которые мы будем думать, как про машины Тьюринга.
	\[
	R_1, R_2, \ldots 
	.\] 

	Построим следующий язык $ \K = \{x \mid x \in \SAT \wedge f(\lvert x \rvert ) \equiv 0 \pmod 2 \}$, где $ f(n)$ ведет себя следующим образом:
	\begin{enumerate}
		\item за $ n$ шагов вычисляем $f(0), f(1), \ldots f(i) \eqqcolon k$ ($ k$ --- последнее значение, которое мы успели вычислить).
		\item тоже за $ n$ шагов:
			\begin{enumerate}
				\item если $ k \equiv 0 \pmod 2$, то проверим, что $\exists z\colon  M_{k/2}(z) \ne \K(z)$, и в случае успеха вернем $ k+1$, иначе (или истратили $ n$ шагов) $ k$;
				\item если $ k \equiv 1 \pmod 2$, проверим, что очередное $ R_{\frac{k-1}{2}}$ правильно сводит $ \SAT$ к $ R$: если  $ \exists z\colon \K(R_{\frac{k-1}{2}}(z)) \ne \SAT(z)$, возвращаем $ k+1$, в любом другом случае --- $ k$.
			\end{enumerate} 
			По факту в пункте (а) мы проверяем, что рассматриваемая машина Тьюринга не решает задачу $\K$. Такое вычисление точно остановится, т.к. ограничено число шагов, т.е., если нынешняя МТ вдруг решала бы задачу $\K$, то число $k$ просто и вернулось бы.
			Здесь первый пункт проверяет $ \K \in \P$, а второй --- $ \SAT \to \K$, то есть $ \K $ $\NP\text{-трудная}$.

			Для какого-то огромного $ n$ найдутся контр-примеры и мы получим $ k+1$. 
		\item $ f(0) = 0$
	\end{enumerate} 
	
	Заметим, что $ \K \in \NP$, так как выполняющий набор можно проверить принадлежность $ \SAT$ за полином и подставить в $ f$, которая тоже работает полином ($ 2n$ шагов).

	\begin{itemize}
		\item
	Пусть $ \K \in \P$, тогда  есть полиномиальная машина $ M$, которая принимает этот язык. Поэтому в пункте (a) мы дальше этой машины не пройдем, так как  контр-примера там нет. То есть с некоторого момента $ f(n) \equiv 0 \pmod 2$, по определению с некоторого места   $ \K = \SAT$, кроме конечного числа случаев, их можно разобрать отдельно. 
	Тогда $ \K \in \NPcomp $ и $ \K \in \P$, поэтому $ \P = \NP$.
\item Если $ \K \in \NPcomp$, то с некоторого момента мы не пройдем пункт (b), так как у нас действительно будет сводимость к $ \K$.
	С некоторого места  $ f(n) \equiv 1 \pmod 2$, а тогда  $ \lvert \K \rvert < \infty$. Следовательно, $ \K \in \P$. Опять противоречие. 
	\end{itemize}
\end{proof}

\section{Полиномиальная иерархия}
\begin{name}
Пусть есть два класса языков $ \Cclass, \Dclass$. Можно построить класс $ \Cclass ^{\Dclass}$, который состоит из  языков вида $ C^{D} $, где $ D \in \Dclass$ и $ C$ --- \underline{машина} для языка из $ \Cclass$.
\end{name}
\begin{defn}[Класс дополнений]\index{класс дополнений}
    \[
    \coC = \{L \mid \overline{L} \in \Cclass\}
    .\] 
\end{defn}
\begin{ex}
	$ \SAT \in \NP$, дополнение к $ \SAT$, то есть всюду ложные формулы (или записи, которые вообще формулами не являются, но их легко проверить полиномиально) $ \in \coNP$.
\end{ex}

\subsection{Полиномиальная иерархия}\index{полиномиальная иерархия}
Самый нижний класс иерархии --- $ \P$. Остальные классы строятся также из  $ \NP$ и $ \coNP$.
\begin{figure}[ht!]
    \centering
    \incfig{hierarchy}
    \caption{Полиномиальная иерархия}
    \label{fig:hierarchy}
\end{figure}
\[
\begin{aligned}
	&\SIGMA ^{0}\P = \PI ^{0}\P = \DELTA ^{0} \P = \P \\
	& \SIGMA ^{i+1} \P = \NP^{\PI^{i}\P} \\
	& \PI^{i+1}\P = \coNP^{\SIGMA ^{i}\P}\\
	& \DELTA ^{i+1}\P = \P^{\SIGMA ^{i}\P} \\
	& \PH = \bigcup_{i \ge 0} \SIGMA ^{i} \P
\end{aligned}\index{\PH}\index{$\SIGMA^{i}\P$}\index{$\PI^i\P$ }\index{$\Delta ^{i}\P$ }
\]

\begin{lm}
    $ \NP^{\PI^{i}\P} = \NP^{\SIGMA^{i}\P}$
\end{lm}
\begin{proof}
    Пусть есть машина $ M$ с оракулом $ A$. Рассмотрим оракул $ \overline{A}$ и построим машину $ M'$, чтобы $ M'^{\overline{A}}$ вела себя аналогично $ M^{A}$. Для этого просто $ M'$ будет переворачивать любой ответ, полученный от оракула, все остальные действия повторяем за $ M$.

	Теперь докажем по индукции утверждение леммы, база очевидна. Переход такой: можно поменять оракула $\PI^{i}\P$ на оракула $\overline{\PI^{i}\P} = \NP^{\SIGMA^{i-1}\P} = \NP^{\PI^{i-1}\P} = \SIGMA^{i}\P$.
\end{proof}

\begin{thm}
    $ L \in \SIGMA^{k}\P$, согда\footnote{тогда и только тогда, когда} существует полиномиально ограниченное отношение $ R \in \PI^{k-1}\P$ такое, что для всех $ x$ :
	\[
		x \in L \Longleftrightarrow \exists y \colon R(x, y)
	.\] 
\end{thm}
\begin{proof}
   \begin{description}
       \item \boxed{ \Rightarrow } 
		   Докажем по индукции. 
		   \begin{itemize}
			   \item \textbf{База:} по определению $ \SIGMA^{1}\P = \NP^\P = \NP$.
			   \item \textbf{Переход:} $ k-1 \to  k$. Пусть $ L = L(M^{O})$, где $ M$ --- полиномиальная НМТ, $ O \in \SIGMA^{k-1}\P$.

				   По предположению индукции для $ O$ существует полиномиально ограниченное $ S \in \PI^{k-2}\P$ такое, что $ \forall q\colon q \in O \Longleftrightarrow \exists w \colon S(q, w)$.

				   Сконструируем из этого $ R$:
				   \begin{itemize}
					   \item $ R(x, y) = 1$, если $ y$ --- принимающая ветвь вычисления $ M^{O}$ (то есть последовательность $0, 1$ -- идти налево или направо), при этом положительные ответы оракула должны быть снабжены сертификатами  $ w\colon  S(q, w) = 1$.

						   То есть все переходы, основанные на ответе оракула <<да>>, должны дополнительно содержать <<доказательство>>.

						   Проверим, что это отношение из $ \PI^{k-1}\P$, и, что оно задает язык $ L$.
					   \item $ R \in \PI^{k-1}\P$: детерминировано проверяем корректность $ y$, а далее проверяем, что ответы оракула были верными. 
					   
					       Для ответов <<нет>> нужно проверить, что $ O$ для него равно нулю (запускаем $ \PI^{k-1}\P$ вычисление), а для ответов <<да>>  --- что  $ S$ равно  $ 1$, то есть проверить сертификат ($ \PI^{k-2}\P$ вычисление).

						   Нужно, чтобы все $ \PI^{k-1}\P$ и $ \PI^{k-2}\P$ (это частный случай $ k-1$) вычисления вернули <<да>>. Для этого построим схему: присоединим к большой конъюнкции все вычисления, чтобы ответ был положительным, нужно, чтобы все ветки $ \PI^{k-1}\P$ (то есть $ \coNP$) вернули одно и то же, при этом мы остаемся в $ \PI^{k}\P$.
				   \end{itemize}
				   Для машины $ M$ существует принимающее вычисление и оно может быть дано в качестве $ y$.

				   Наоборот, если у нас есть корректное вычисление машины $ M$, то оно и будет принадлежать $ L$.
		   \end{itemize}
       \item \boxed{ \Leftarrow }
		   Пусть у нас есть отношение $ R$.
		   Возьмем машину с оракулом $ R$.
		   
		   Она будет недетерминировано 
		   выбирать $ y$ и проверять, то есть спрашивать у оракула, $ R(x, y)$.
		   
		   Эта машина и будет распознавать наш язык $L$.
   \end{description}  
\end{proof}

Аналогично можно доказать следующую теорему
\begin{thm}
	$ L \in \PI^{k}\P$, согда существует полиномиально ограниченное отношение $ R \in \SIGMA^{k-1}\P$ такое, что для всех $ x$ :
	\[
		x \in L \Longleftrightarrow \forall  y \colon R(x, y)
	.\] 
\end{thm}
\begin{cor}\label{cor:polysigkpi}
    $ L \in \SIGMA^{k}\P$, согда существует полиномиально ограниченное отношение $ R \in \P$ такое, что для всех $ x$:
	\[
		x \in L \Longleftrightarrow \exists y_1 \forall y_2 \exists y_3 \ldots R(x, y_1, y_2, \ldots y_k)
	.\] 
\end{cor}
\begin{cor}
    $ L \in \PI^{k}\P$, согда существует полиномиально ограниченное отношение $ R \in \P$ такое, что для всех $ x$:
	\[
		x \in L \Longleftrightarrow \forall y_1 \exists  y_2 \forall  y_3 \ldots R(x, y_1, y_2, \ldots y_k)
	.\] 
\end{cor}

