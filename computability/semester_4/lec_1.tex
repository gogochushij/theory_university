\chapter{Вычислимость. Система вычислимости по Клини} 
\section{Рекурсивные функции}

\lecture{1}{11 feb}{\dag}
\begin{defn}[]\index{$ k$-местная частичная функция}
	Пусть функция $ f\colon \N^{k} \to  \N, ~ k \in \N$, где $ \N = \{0, 1, 2, \ldots \}$. Такая функция называется \selectedFont{$ k $-местной частичной функцией}. Если $ k = 0$, то $ f = \const$.
\end{defn}

\subsection{Простейшие функции} \index{простейшие функции}
\selectedFont{Простейшими} будем называть следующие функции:
\begin{itemize}
	\item Нуль местный нуль --- функция без аргументов, возвращающая $ 0$;
	\item Одноместный нуль --- $ 0(x) = 0$;
	\item Функция следования --- $ s(x) = x + 1$;
	\item Функция выбора (проекция) ---  $ I_{n}^{m}(x_1, \ldots x_{n}) = x_m$
\end{itemize}


\subsection{Операторы}
Определим три оператора:
\begin{defn}
\begin{itemize}
    \item \index{оператор суперпозиции}
	Функция $ f$ \selectedFont{получается оператором суперпозиции} из функций $ h$ и $ g_i$, где
	\[
		h(y_1, \ldots , y_m), ~ g_i(x_1, \ldots , x_n); ~ 1 \le i \le n
	,\] 
	если 
	\[
		f = h(g_1(x_1, \ldots, x_{n}), \ldots g_m(x_1, \ldots , x_{n}))
	.\] 
	Оператор обозначается $\S$.
\item \index{оператор примитивной рекурсии}
	Функция $ f^{(n+1)} $\footnote{Здесь и далее $ f^{(n)}$ обозначается функция, принимающая $ n$ аргументов, то есть $ n$-местная}
	\selectedFont{получается оператором примитивной рекурсией} из $ g^{(n)}$  и $ h^{(n+2)}$, если 
	\[
	\begin{cases}
		f(x_1, \ldots x_{n}, 0) = g(x_1, \ldots x_{n}) \\
		f(x_1, \ldots x_{n}, y+1) = h(x_1, \ldots x_{n}, y, f(x_1, \ldots x_{n}, y))
	\end{cases}
	\] 
	Оператор обозначается $ \R$.
\item \index{оператор минимизации}
	Функция $ f$ задается \selectedFont{оператором минимизации} ($ \M$), если она получается из функции  $ g$ следующим образом:
	\[
	\begin{aligned}
		f(x_1, \ldots x_{n}) & = \mu y \bigl[ g(x_1, \ldots x_{n}, y) = 0 \bigr] = \\
							 &= 
							 \begin{cases}
								 y & g(x_1, \ldots x_{n}, y) = 0 \wedge g(x_1, \ldots x_{n}, i)\footnote{подразумевается, что функция определена в этих точках} \ne 0 ~ \forall i < y \\
								 \uparrow\footnote{не определена} & else
							 \end{cases}
	\end{aligned}
	\]
\end{itemize}
\end{defn}

\begin{ex}
    \[
    x - y = \begin{cases}
		x - y, & x \ge  y \\
		\uparrow, & x < y
    \end{cases}
    \] 
	Можно задать, используя оператор минимизации:
	\[
		x - y = \mu z [\lvert (y+z) - x \rvert = 0]
	.\] 
\end{ex}



\subsection{Функции}
\begin{defn}[Примитивно рекурсивная функция]\index{примитивно рекурсивная функция}
	Функция $ f$ называется \selectedFont{примитивно рекурсивной} (\prf),
	если 
	существует последовательность таких функций  $ f_1, \ldots f_k$, что
	все $ f_i$ либо простейшие, либо получены из предыдущих $ f_1, \ldots f_{i-1}$ с помощью одного из операторов $\S$ и $\R$ и $ f = f_k$.
\end{defn}


\begin{ex}\label{ex:1}
	Докажем, что $ f(x, y) = x + y$ --- \prf. По  $ \R$ можем получить $ f$ так:
	\[
	\begin{cases}
		f(x, 0) &= x = I^{1}_{1} (x) =g \\
		f(x, y+1) &= (x + y) + 1 = s(f(x, y)) = s(I^{3}_{3} (x, y, f(x, y)) = h 
	\end{cases}
	\] 
	Теперь построим последовательность функций $ f_i$, где последним элементом будет $ f$, полученный с помощью $ \R$: 
	\[
		I^{1}_{1} , ~s, ~I^{3}_{3} , ~ h = \S(s, I^{3}_{3}) ,~ f
	.\] 
\end{ex}


\begin{defn}[Частично рекурсивная функция]\index{частично рекурсивная функция}
	Функция $ f$ называется \selectedFont{частично рекурсивной функцией} (\crf), если существует последовательность функций $ f_1, \ldots f_k$, таких что $ f_i$ либо простейшая, либо получается из предыдущих с помощью одного из операторов  $ \S, \R, \M$.
\end{defn}

\begin{note}
    Частично рекурсивная функция может быть не везде определена. Но примитивно рекурсивная определена везде.
\end{note}
\begin{note}
    Существуют частично рекурсивные функции, которые всюду определены, но при этом не являются \prf.
\end{note}

\begin{defn}\index{общерекурсивная функция}
	\selectedFont{Общерекурсивная функция} --- всюду определенная частично рекурсивная.
\end{defn}


\begin{ex}
	$ \mu y [x + y + 1 = 0]$ --- нигде не определена, но получается из последовательности других функций с помощью операторов.
\end{ex}



\begin{lm}\label{lm:rec}
    Следующие функции являются \prf:
	\begin{enumerate}
	\item $ \const ^{(n)}$ 
	\item $ x + y$ 
	\item $ x\cdot y$ 
	\item $ x ^{y}$, где $ 0^{0}$ можем определить, как хотим
	\item $ \sg(x) = \begin{cases}
			0 & x= 0 \\
			1 & x\ne 0
		\end{cases}$
	\item $ \bsg(x) = \begin{cases}
			1 & x= 0 \\
			0 & x\ne 0
		\end{cases}$
	\item $ x \dotminus 1 = \begin{cases}
			u & x = 0 \\
			x - 1 & x > 0
		\end{cases}$ 
	\item $ x \dotminus y = \begin{cases}
			0 & x < y \\
			x - y & else
		\end{cases}$ 
	\item $ \lvert x - y \rvert $
	\end{enumerate}
\end{lm}
\begin{proof*}
	\begin{enumerate}
		\item Сначала можем получить нужное число последовательной суперпозицией функции следования (получили константу от одной переменной), затем проецируем $ I^{n + 1}_{1}$, чтобы получить $ n$ переменных (первая - наша константа).
		\item Доказали выше \hyperref[ex:1]{в примере \ref{ex:1}}.
		\item $ f(x, y) = xy$ определим так:
			\[
			\begin{cases}
				f(x, 0) &= 0\\
				f(x, y+1) &= f(x, y) + x
			\end{cases}
			\] 
			а складывать мы умеем.
		\item $ f(x, y) = x^{y}$ :
			\[
			\begin{cases}
				f(x, 0) &= 1 = s(0) \\
				f(x, y+1) &= f(x, y) * y
			\end{cases}
			\] 
			Умножать тоже можно по третьему пункту.
		\item  $ \sg(x) = \begin{cases}
				1&x=0 \\
				0& x\ne 0
		\end{cases}$
		\[
		\begin{cases}
			\sg(0) &= 0 \\
			\sg(x+1) &= 1 = s(0)
		\end{cases}
		\] 
	\item Аналогично
	\item $ f(x) = x \dotminus 1$ 
		\[
		\begin{cases}
			f(0) &=0 \\
			f(x+1) &= x = I^{1}_{1} (x)
		\end{cases}
		\] 
	\item $ f(x, y) = x \dotminus y$
		\[
		\begin{cases}
			f(x, 0) &= x = I^{1}_{1}(x)  \\
			f(x, y+1) &= f(x, y) \dotminus 1
		\end{cases}
		\] 
	\item $ f(x, y) = \lvert x - y \rvert  = (x \dotminus y) + (y \dotminus x)$
    \end{enumerate}
\end{proof*}
\begin{note}
    Обычное вычитание не является \prf, так как не везде определено на $ \N$.
\end{note}



\subsection{Оператор ограниченной минимизации}
\begin{defn}[Оператор ограниченной минимизации]\index{оператор ограниченной минимизации}
	Функция $ f^{(n)}$ задается \selectedFont{оператором ограниченной минимизации} из функций $ g^{(n+1)}$ и $ h^{(n)}$, если
	 \[
		\mu y \le h(\overline{x}) \bigl[ g(\overline{x}, y) = 0\bigr]\footnote{Здесь и далее $ \overline{x} = x_1, \ldots x_{n}$.} \\
	 .\]
	 Это означает, что
	 \[
		 f(\overline{x}) = 
		 \begin{cases}
			 y & g(\overline{x}, y) = 0 \wedge y \le h(\overline{x}) \wedge g(\overline{x}, i) \ne 0\footnote{Аналогично, подразумевается, что функция определена в этих точках} ~ \forall i < y \\
			 h(\overline{x})+1 & else
		 \end{cases}
	 \] 
\end{defn}


\begin{st}
    Пусть $ g^{(n+1)}, h^{(n)}$ --- примитивно рекурсивные функции, $ f^{(n)}$ получается из $ g$ и $ h$ с помощью ограниченной минимизации, то $ f$ тоже \prf.
\end{st}
\begin{proof*}
	Заметим, что $ f$ можно получить следующим образом:
	\[
		f(\overline{x}) = \sum_{y=0}^{h(x)}	\prod_{i=0}^{y}\sg(g(\overline{x}, i))
	.\] 
	Внутреннее произведение равно единице только тогда, когда все $ g(\overline{x}, i) \ne 0$. Если для некоторого $ y$ обнуляется $ g(\overline{x}, y)$, то все произведения, начиная с $ y+1$, будут равны нулю, поэтому просуммируются только  $ y$ единиц. 
	Если же такого $ y$ нет, получим сумму из $ h(\overline{x}) + 1$ единицы. Именно это и нужно.

	Проверим, что можно получить $ a(\overline{x}, y) = \sum_{i=0}^{y} g(\overline{x}, i)$ и $ m(\overline{x}, y) = \prod{i=0}^{y} g(\overline{x}, i)$ с помощью примитивной рекурсии:
	\[
	\begin{aligned}
		&\begin{cases}
			a(\overline{x}, 0) &= g(\overline{x}, 0) \\
			a(\overline{x}, y+1) &= a(\overline{x}, y) + g(\overline{x}, y+1)
		\end{cases}
		&\begin{cases}
			m(\overline{x}, 0) &= g(\overline{x}, 0) \\
			m(\overline{x}, y+1) &= m(\overline{x}, y) \cdot g(\overline{x}, y+1)
		\end{cases}
	\end{aligned}
	\]
\end{proof*}


\begin{note}
	$ {\color{red}0(x)}$ можно исключить из определения простейших функций, так как она получается с помощью $ \R$ для нульмерного $ {\color{green}0}$ и $ I^{2}_{2} (x, y)$ :
	\[
		{\color{red}0(y)} = 
		\begin{cases}
			{\color{red}0(0)} &= {\color{green}0} \\
			{\color{red}0(y+1)} &=I^{2}_{2} (y, {\color{green}0})
		\end{cases}
	\] 
\end{note}


\subsection{Предикаты}
\begin{defn}\index{предикаты}
	Предикат --- условие задающее подмножество: $ R \subset \N^{k}$.
	
	\noindent
	Предикат называется \selectedFont{примитивно рекурсивным (общерекурсивным)}, его характеристическая функция примитивно рекурсивная (общерекурсивная).
\[
	\chi_{R}(\overline{x})= 
	\begin{cases}
		1, &\overline{x} \in  R \\
		0, &\overline{x} \notin R
	\end{cases}
\] 
\end{defn}


\begin{st}
	~\begin{itemize}
		\item Если  $ R, Q$ --- примитивно рекурсивные (общерекурсивные) предикаты, то $ P \vee Q, ~ P \wedge Q, ~ P \to Q, \neg P$ тоже примитивно рекурсивные (общерекурсивные).
		\item Предикаты $ = , ~\le , ~\ge , ~<, ~>$ тоже примитивно и общерекурсивны.
    \end{itemize}
\end{st}
\begin{proof*}
	~\begin{itemize}
		\item Проверим, что характеристические функции примитивно / общерекурсивны: 
			\[
			\begin{aligned}
				\chi_{P\vee Q}(\overline{x}) & = \chi_{P}(\overline{x}) \cdot \chi_{Q}(\overline{x}) \\
				\chi_{P \wedge Q}(\overline{x}) & = \sg(\chi_{P}(\overline{x}) + \chi_{Q}(\overline{x})) \\
				\chi_{P \to Q}(\overline{x}) &= \bsg(\chi_{P}(\overline{x}) + \bsg(\chi_{Q}(\overline{x}))) \\
				\chi_{\neg P} (\overline{x}) & = \bsg(\chi_{P}(\overline{x}))
			\end{aligned}
			\]
		\item Аналогично выразим, через простейшие:  
			\[
			\begin{aligned}
				\chi_{x=y} (x) &= \bsg(\lvert x - y \rvert ) = \begin{cases}
					1, &x=y \\
					0, &x\ne y
				\end{cases} \\
					\chi_{x< y}(x) &= \sg(x \dotminus y)
			\end{aligned}
			\]
			Остальные можем выразить также или через уже проверенные $ <$ и $ \neg$.
    \end{itemize}
\end{proof*}


\begin{lm}
    Следующие функции являются примитивно рекурсивными:
	\begin{enumerate}
		\item $ \left\lfloor \frac{x}{y} \right\rfloor$, считаем, что $ \left\lfloor \frac{x}{0} \right\rfloor = x$
		\item $\Div(x, y) = 
\begin{cases}
	1, & y \mid x \\
	0, & else
\end{cases}$
\item $ \Prime(x) = \begin{cases}
		1, & x \in \Pm\\
		0, & else
\end{cases}$
\item  $ f(x) = p_{x}$, где $ p_{x} $ --- $ x$-тое простое число, $ p_0 \coloneqq  2$
\item  $ \expp(i, x) $ --- степень простого числа $ p_i$ разложении $ x$, $ \expp(i, 0) \coloneqq 0$
	\end{enumerate}
\end{lm}
\begin{proof*}
    \begin{enumerate}
		\item $ f(x, y) = \left\lfloor \frac{x}{y} \right\rfloor $. Найдем минимальное $ k$, что $ f'(x, y, k) = yk > x$. Чтобы получить  $ f(x, y) = \min(k \mid f'(x, y, k)) - 1$. Используем оператор минимизации:
			 \[
				 f(x, y) = \mu k [ \neg f'(x, y, k) = 0] - 1
			.\] 
		\item $ \Div(x, y) = \left\lfloor \frac{x}{y} \right\rfloor \cdot y = x$
		\item Определим  $ \Div'(x, y) = (y \le  1) \vee (\neg\Div(x, y))$, эта функция проверяет, что $ y$ не является нетривиальным делителем $ x$.

			Теперь, используя ограниченную минимизацию, выразим  $ \Prime(x)$ :
			\[
				\Prime(x) = \Bigl(\mu y \le h(x) [\Div'(x, y) = 0]\Bigr) = x, \text{ где } h(x) = x - 1 
				.\]
				То есть мы посмотрели на все меньшие числа, если среди них найдется нетривиальный делитель, то число не простое.
			\item Пусть $ f'(x) = \text{ количество простых } \le x $. 
				\[
					\begin{cases}
						f'(0) &= 0 \\
						f'(x+1) &= \Prime(x+1) + f(x)
					\end{cases}
				\] 
				Теперь можно вычислить $ f(x)$: для этого определим функцию $ g(x, y) = (f'(y) = x)$,
				 \[
					 f(x) = \mu y [ \neg f'(x, y)  = 0] 
				.\] 
			\item Чтобы найти степень вхождения простого числа $ p_i$ в $ x$, сначала находим это простое число по номеру, затем находим минимальное $ k$, что $ x$ не делится на $ p_i^{k}$ и вычитаем единицу.
    \end{enumerate} 
\end{proof*}


\subsection{Теоремы про рекурсии}
\begin{thm}[Канторовская нумерация]
    Пусть $ \pi\colon \N \times \N \to  \N$ :
	\[
		\pi(x, y) = \frac{1}{2}(x+y) (x+y+1)+y
	.\] 
	\begin{itemize}
	\item
	Тогда для любого $ z$ существует единственное представление $ z = \pi(x, y)$.
\item Причем функции $ x(z), y(z)$ примитивно рекурсивные.
	\end{itemize}
\end{thm}
\begin{proof*}
	\begin{itemize}
		\item
	Запишем $ \pi(x, y) = {{x+y}\choose{2}} + y$. Заметим, что для $ n > m$ верно  
	$$ {n \choose 2} - {m \choose 2} \ge {n \choose 2} - {n-1 \choose 2} = n-1.$$
	Предположим, что $ x + y > x' + y'$ и  $ \pi(x, y) = \pi(x',y')$. Тогда
	 \[
		 y' - y = {x+y \choose 2} - {x'+y' \choose 2} \ge x + y - 1 \ge x' + y'
	.\] 
	Но $ y \ge 0, ~ x' \ge 0$,  поэтому единственный возможный вариант, когда они равны нулю, а $ x+y = x'+y'+1$. Проверим на равенство  $ \pi(x, y)$ и $ \pi(x', y')$:
	\[
	\begin{aligned}
		\pi(x, y) & = \frac{1}{2}x(x+1) = \frac{1}{2}(y'+1)(y'+2) = \frac{1}{2}y'(y'+1) + y' + 1 \\
		\pi(x', y') &= \frac{1}{2}y'(y'+1) + y'
	\end{aligned}
	\]
	Равенства нет.
\item  Можно по-честному все посчитать и выразить $ x(z), ~y(z)$.
	Пусть 
	\[
	\begin{aligned}
		w &= x+y \\
		t &= \frac{1}{2}w(w+1) = \frac{w^2+w}{2} \\
		z &= t +y
	\end{aligned}
	\]
	Решим квадратное уравнение, чтобы выразить $ w$ через $ t$ (отрицательный корень можем сразу отбросить):
	\[
		w = \frac{-1 + \sqrt{ 8t + 1}}{2}
	.\] 
	Запишем неравенство:
	\[
		t \le z = t + y < t + (w +1) = \frac{(w+1)^2+(w+1)}{2}
	.\] 
	Отсюда
	\[
		w \le \frac{-1  + \sqrt{ 8z +1} }{2} < w+1
	.\] 
	Тогда 
	\[
	\begin{aligned}
		w &= \left\lfloor \frac{-1 + \sqrt{ 8z+1} }{2} \right\rfloor \\
		t &= \frac{w^2+w}{2} \\
		y &= z - t \\
		x &= w - y
	\end{aligned}
	\]
	Таким образом, мы выразили через $ z$ обе координаты. Единственный момент --- нужно извлекать корень, в натуральную степень возводить мы умеем, поэтому можем с помощью ограниченной минимизации перебрать все меньшие числа, возвести их в квадрат и сравнить с нашим числом.
	\end{itemize}
\end{proof*}


\begin{thm}[Возвратная рекурсия]\index{рекурсия возвратная}
    Зафиксируем $ s$. Пусть 
	\[
	\begin{cases}
		f(\overline{x}, 0) &= g(\overline{x}) \\
		f(\overline{x}, y+1) &= h(\overline{x}, y, f(\overline{x}, t_1(y)), \ldots f(\overline{x}, t_{s}(y)))
	\end{cases}
	\] 
	где $ \forall 1 \le i \le s ~t_i(y) \le y$, $ g^{(n)}, h^{(n+1+s)}, t_i^{(1)}, $.

	\noindent
	Тогда, если $ g, h, t_i$ --- примитивно / общерекурсивные, то и $ f$ тоже.
\end{thm}
Основная идея этой теоремы --- можем использовать все ранее вычисленные значения функции, а не только предыдущее.
\begin{proof*}
	Построим с помощью примитивной рекурсии функцию $ m(\overline{x}, y)$, которая возвращает закодированную последовательность $ f(\overline{x}, i), ~ 0 \le i \le y$.

	Кодировать будем так: каждому $ f(\overline{x}, i)$ будет соответствовать $ p_i$ ($ i$-ое простое число) в степени $ 1 + f(\overline{x}, i)$. 

	Если мы построим эту функцию, то $ f(\overline{x}, y)$ --- уменьшенная на $ 1$ степень $ y$-ого простого, обозначим функцию, которая это делает:
	\[
		f(\overline{x}, y) = \ith(y, m(\overline{x}, y))
	.\] 
	Вернемся к построению $ m$:
	\[
	\begin{cases}
		m(\overline{x}, 0) &= 2^{1+g(\overline{x})} \\
		m(\overline{x}, y+1) &= m(\overline{x}, y) \cdot p_{y+1}^{1+h\left( 
			\overline{x}, y, \ith(t_1(y), m(\overline{x}, y)), \ldots \ith(t_{k}(y), m(\overline{x}, y))
	\right)
}
	\end{cases}
	\] 
\end{proof*}


\begin{thm}[Совместная рекурсия]\index{рекурсия совместная}
	Пусть $ f_i^{(n+1)}, ~ 1 \le i \le k$,
	\[
	\begin{cases}
		f_i(\overline{x}, 0) &= g_i(\overline{x}) \\
		f_i(\overline{x}, y+1) &= h_i (\overline{x}, y, f_1(\overline{x}, y), \ldots f_{k}(\overline{x}, y))
	\end{cases}
	\] 
	Если $ g_i^{(n)}, h_i^{(k+2)}, ~1 \le i \le k$ --- примитивно / общерекурсивные, то $ f_i$ тоже.
\end{thm}
Основная идея этой теоремы --- можем использовать $ y$-е значение каждой из $ k$ функций.
\begin{proof*}
	Заметим, что канторовскую функцию можно, последовательно применив несколько раз, расширить до $ k$-местной. Обозначим полученную функцию за $ c$, а обратные за $ c_1, \ldots c_{k}$.

	Давайте просто объединим все $ f_i$ в одну функцию 
	\[
		m (\overline{x}, y) = c\left(f_1(\overline{x}, y), \ldots f_k(\overline{x}, y)\right)
	.\]
	Теперь каждую $ f_i$  можно вычислить
	\[
		f_i(\overline{x}, y) = c_i(m (\overline{x}, y))
	.\] 
	Чтобы получить $ m$ достаточно использовать примитивную рекурсию:
	\[
	\begin{cases}
		m(\overline{x}, 0) &= c\left(g_1(\overline{x}), \ldots g_k(\overline{x})\right) \\
		m(\overline{x}, y+1) &= c ( \\
							 &\quad\begin{aligned}
				&h_1\left( \overline{x}, y, c_1(m(\overline{x}, y)), \ldots c_k(m(\overline{x}, y)) \right), \\
				&\quad\vdots \\
				&h_k\left( \overline{x}, y, c_1(m(\overline{x}, y)), \ldots c_k(m(\overline{x}, y)) \right)
			\end{aligned} \\
							 &) 
	\end{cases}
	\] 
\end{proof*}


\begin{thm}[Кусочное задание функции]\index{кусочное задание функции}
	Пусть $ R_0, \ldots R_k $ --- отношения\footnote{Набор предикатов}, такие что $ \bigsqcup\limits_{i=0}^{k} R_i = \N^{m}$
	\footnote{То есть для $ i \ne  j$ верно $ R_{i} \cap R_j = \varnothing$.}.
	Для $ \lvert \overline{x} \rvert = n$ кусочно зададим функцию $ f^{(n)}$:
\[
	f(\overline{x}) = \begin{cases}
		f_0(\overline{x}), & \text{если }R_0(\overline{x}) \\
		f_1(\overline{x}), &  \text{если }R_1(\overline{x}) \\
		\quad\vdots & \quad\vdots \\
		f_k(\overline{x}), &  \text{если }R_k(\overline{x}) \\
	\end{cases}
\] 
Если $ f_i^{(n)}, R_i$ --- примитивно / общерекурсивны, то и $ f$ тоже.
\end{thm}
\begin{proof*}
	Рассмотрим характеристические функции $ \chi_{R_{i}} $ для $ R_i$.
	Тогда 
	\[
		f(\overline{x}) = \sum_{i=0}^{k} f_i(\overline{x}) \cdot \chi_{R_{i}}(\overline{x})
	.\] 
	А это просто сумма произведений, которые мы можем вычислять.
\end{proof*}


\section{Равносильность МТ и \crf}
\begin{thm}
	Функция вычисляется машиной Тьюринга тогда и только тогда, когда она частично рекурсивная (то есть вычислима по Клини).
\end{thm}
\begin{proof}
    \begin{description}
        \item \boxed{ 2 \Longrightarrow 1} 
			Если $ f(x_1, \ldots x_n) = y$, то считаем, что МТ получаем $ 1^{x_1}01^{x_2}0\ldots 01^{x_n}$ и должна выдать $ 1^{y}$; если  $ f$ не определена, МТ должна зацикливаться и наоборот.
			\begin{itemize}
				\item Для простых функций можем построить МТ напрямую:
					\begin{itemize}
						\item Если мы хотим выдавать нуль, просто стираем вход.
						\item Если нужно увеличить число на один, приписываем $ 1$ в конец справа.
						\item Если нужно вернуть $ k$-ую проекцию, стираем все до начала $ k$-ого числа (то есть нужно отсчитать $ k-1$ нуль на входе), далее стереть все после.
					\end{itemize}
				\item Для операторов $ \S, \R, \M$:
					\begin{description}
						\item[$ \S$:]
							Пусть есть набор функций $ h^{(n)}, ~ g_1^{(m)}, \ldots , ~ g_n^{(m)} \longrightarrow f^{(m)}$, для каждой из которых есть машина Тьюринга $ M_{h}$ и $ M_{g_{i}}$. 

							Хотим построить МТ $ M_{S}$ для $ S$.

							Сделаем это так:
							\begin{itemize}
								\item Копируем весь вход $ n$ раз:
									\[
										\Bigl(1^{x_1}01^{x_2}\ldots 01^{x_n} \ast \Bigr)^{n}
									.\] 
								\item Запускаем $ M_{g_i}$ на соответствующей части полученного входа. 

							Если нужно что-то записать, то будем сдвигать всю правую часть на нужное число клеток, чтобы освободить для место.

							МТ запускаем  псведопараллельно (по очереди даем поработать).

							В каждой часть после окончания работы оставляем только ответ:
							\[
								1^{y_1}\ast 1^{y_2}\ldots \ast1^{y_n}
							,\] 
							где $ y_i = g_i(x_1, \ldots x_m)$.
						\item Запускаем на этом результате $ M_{h}$.
							\end{itemize}

\lecture{2}{18 feb}{\dag}
						\item[$ \R$:]
							Пусть рекурсия задает $ f^{(m+1)} (x_1, \ldots x_m, y)$ из $ g^{(m)}$ и $ h^{(m+2)}$.
							\[
							\begin{cases}
								f(\overline{x}, 0) & = g(\overline{x}) \\
								f(\overline{x}, y+1) &= h(\overline{x}, y, f(\overline{x}, y))
							\end{cases}
							\] 
							Считаем, что для $ g, h$ уже есть МТ ($M_{g}$ и $M _{h}$), и мы хотим построить $ M_{f}$, которая будет вычислять $ f$.

							Построим вспомогательные МТ:
							\begin{itemize}
								\item $ M_1$:  для входа $ 1^{x_1}0\ldots 01^{x_m}01^{y}$ построим $ 1^{y}01^{x_1}0\ldots 01^{x_m}001^{g(x_1, \ldots x_m)}$. Для этого просто запустим $ M_ g$ на входе, но не будем стирать его, а результат просто припишем после двух нулей справа.
								\item $ M_2$:  для входа $ 1^{y}01^{x_1}0\ldots 01^{x_{m}}01^{u}01^{z}$ построим $ 1^{y}01^{x_1}0\ldots 01^{x_m}01^{u+1}01^{h(x_1, \ldots x_{m}, u, z)}$, аналогично, используя $ M_{h}$, допишем в конец вместо $ z$ результат $ h$ и допишем единицу к $ 1^{u}$. Здесь  $ u+1$ обозначает текущее значение $ y'$, а значение $ h$ --- значение $ f(y')$. 
								\item $ M_3$:  для входа $ 1^{y}01^{x_1}0\ldots 01^{x_{m}}01^{u}0^{z}$ оставим только $ 1^{z}$.
								\item $ \Phi $: для входа $ 1^{y}01^{x_1}0\ldots 01^{x_{m}}01^{u}01^{z}$ проверим, что $ u \ne y$. 
							\end{itemize}
							Теперь соберем все вместе: сначала запустим $ M_1$, далее пока $ \Phi $ возвращает неравенство, запускаем $ M_2$ (увеличиваем $ u$ на один, вычисляем следующее значение функции), и в конце стираем лишнее, запустив $ M_3$.
						\item[$ \M$:] Хотим по МТ $ M_{g}$ построить $ M_{f}$, вычисляющую 
							\[
								f(\overline{x}, y) = \begin{cases}
									y & g(\overline{x}, y) = 0 \wedge g(\overline{x}, z) \ne  0 \quad \forall z < y \\
									\uparrow & else
								\end{cases}
							\] 
							Аналогично построим несколько вспомогательных МТ:
							\begin{itemize}
								\item $ N_1$ :  приписывает $ 0$ ко входу:
									\[
									1^{x_1}0 \ldots 01^{x_{m}} \longrightarrow 1^{x_1}0\ldots 01^{x_{m}}0
									.\] 
								\item $ N_2$ :  дублирует вход, разделяя решеткой: $ w \longrightarrow w\#w$ 
								\item $ N_3$ :  в продублированному входе меняет вторую половину на результат $ M_{g}$ 
									\[
										1^{x_1}0\ldots 01^{x_{m}}01^{y} \# 1^{x_1}0\ldots 01^{x_{m}}01^{y} \stackrel{M_{g}}{\longrightarrow} 	1^{x_1}0\ldots 01^{x_{m}}01^{y} \# 1^{g(x_1, \ldots x_{m}, y}
									.\] 
								\item $ N_4$ :  очищает все после решетки и дописывает единицу в конец
									\[
1^{x_1}0\ldots 01^{x_{m}}01^{y} \# w \longrightarrow 
1^{x_1}0\ldots 01^{x_{m}}01^{y+1}
									.\] 
								\item $ N_5$ : стирает  все, кроме ответа
									\[
1^{x_1}0\ldots 01^{x_{m}}01^{y} \# w \longrightarrow 1^{y}
									.\] 
								\item $ \Phi $ : проверяет, что после решетки что-то еще есть
									$ w\# v \longrightarrow v \ne \varepsilon $.
							\end{itemize}
							Теперь можем построить $ M_{\mu}$ так: 
							\[
								N_1; ~ N_2; ~ N_3; \operatorname{while} \Phi \operatorname{do} N_4, ~ N_2, ~ N_3; ~ N_5
							.\] 
					\end{description}
			\end{itemize}
        \item \boxed{ 1 \Longrightarrow 2} 
			Теперь мы хотим промоделировать работу МТ с помощью частично рекурсивной функции. На вход должны либо выдать результат, либо зациклится. 
			Так как машины Тьюринга работают со строками, а функции с натуральными числами, нужно придумать правила кодирования.

			Пусть есть конфигурация МТ 
			$$ \alpha q_i a_j \beta ,$$ где $ \alpha $ --- строка слева от головки, $ q_i$ --- состояние, $ a_j$ --- текущий символ, $ \beta $ --- справа от головки.

			Пронумеруем рабочий алфавит $ \Gamma = \{a_0, \ldots a_{m-1}\}$, где $ a_0$ --- пустой символ (\textvisiblespace).

			\begin{description}
				\item[Кодирование конфигураций]
			Теперь можем конфигурацию записать как 
			 \[
				 \widetilde{ \alpha } , \widetilde{ q_i} , \widetilde{ a_j} , \widetilde{ \beta } 
			,\] 
			где $ \widetilde{ \alpha } $ --- число, соответствующее $ \alpha $  в $  m$-ичной записи, $ \widetilde{ q_i} $ --- просто номер состояния, $ \widetilde{ a_j} $ --- номер  в алфавите ($ j$), $ \widetilde{ \beta } $ --- число, соответствующее $ \beta $ в $ m$-ичной записи, записанное справа налево. 

			Сдвиги обозначать будем $ d$: вправо  $d= 1$, влево $ d= 2$.

			Терминальное состояние --- $ z$. 
			Множество состояний тоже пронумеруем и получим множество состояний $ \widetilde{ Q}  = \{0, 1, \ldots \lvert Q \rvert - 1\}$.

			\begin{ex}
			    Рассмотрим небольшой пример. $ \Gamma = \{a_0, a_1\}$, тогда состояние 
				$
					\underbrace{a_1a_0a_1a_1a_0}_{ \alpha } q_3 \underbrace{a_1 a_1 a_0 a_1a_1}_{ \beta }
					$
				будет записано так:
				$
					(22, 3, 1, 13)
					$.
			\end{ex}

		\item [Кодирование команд]
			Пусть есть переход $ (q, a) \to (p, b, d)$. Сопоставим $ p, b, d$ тройку функций $ \varphi _{q}, \varphi _{a}, \varphi _{d}$:
			\[
			\begin{aligned}
				&\varphi _a \colon \widetilde{ Q} \times \widetilde{ \Gamma  }  \to \widetilde{ \Gamma } \\
				& \varphi _{q} \colon \widetilde{ Q} \times \widetilde{ \Gamma }  \to  \widetilde{ Q} \\
				& \varphi _{d} \colon \widetilde{ Q} \times \widetilde{ \Gamma } \to \{1, 2\}
			\end{aligned}
			\]
			Эти функции будут примитивно рекурсивными, так как заданы на конечном множестве, на остальных можем доопределить нулем.
		\item[Преобразование конфигураций]
			Пусть у нас есть переход между двумя конфигурациями: 
			$$ K = \alpha q_i a_j \beta  \to \alpha ' q_i' a_j' b' = K'.$$
			Зададим функцию на числах, которая проделает этот переход $ \Phi \colon K \to  K'$. На самом деле эта функция состоит из четырех, которые мы сейчас и определим.

			Пусть 
			\[
			\begin{aligned}
				&\widetilde{ q_i}'  (\widetilde{ \alpha } , \widetilde{ q_i}, \widetilde{ a_j} , \widetilde{ \beta } ) &&= \varphi_{q}(\widetilde{ q_i} , \widetilde{ a_i} ) \\
				&\widetilde{ \alpha } ' (\widetilde{ \alpha } , \widetilde{ q_i} , \widetilde{ a_j} , \widetilde{ \beta } ) &&= 
				\begin{cases}
					\widetilde{ \alpha } \cdot m + \varphi_{a}(\widetilde{ q_i} , a_j), & \varphi _{d}(\widetilde{ q_i} ,\widetilde{ a_j} ) = 1 \\
					\left\lfloor \frac{\widetilde{ \alpha } }{m} \right\rfloor, & \varphi _{d}(\widetilde{ q_i} ,\widetilde{ a_j} ) = 2
				\end{cases}
					\\
				&\widetilde{ \beta } ' (\widetilde{ \alpha } , \widetilde{ q_i} , \widetilde{ a_j} , \widetilde{ \beta } ) &&= 
				\begin{cases}
					\widetilde{ \beta } \cdot m + \varphi_{a}(\widetilde{ q_i} , a_j), & \varphi _{d}(\widetilde{ q_i} ,\widetilde{ a_j} ) = 1 \\
					\left\lfloor \frac{\widetilde{ \beta } }{m} \right\rfloor, & \varphi _{d}(\widetilde{ q_i} ,\widetilde{ a_j} ) = 2
				\end{cases}
		\\
				& \widetilde{ a_j} ' &&= 
		\begin{cases}
			\widetilde{ \beta }  \mod m, & \varphi _{d}(\widetilde{ q_i} ,\widetilde{ a_j} ) = 1 \\
			\widetilde{ \alpha  }  \mod m, & \varphi _{d}(\widetilde{ q_i} ,\widetilde{ a_j} ) = 2 
		\end{cases}
			\end{aligned}
	\]
	Заметим, что все эти формулы примитивно рекурсивные\footnote{Единственное, чего нет явно в \hyperref[lm:rec]{лемме \ref{lm:rec} выше}, это остаток по модулю, но его легко получить из деления нацело.}.
\item[Общая работа МТ]
	Пусть $ K(0) = \left( \widetilde{ \alpha _0} , \widetilde{ q_0}, \widetilde{ a_0} , \widetilde{ \beta _0} \right) $ --- начальная конфигурация.
	Чтобы получить новую конфигурацию для шага $ t$, посчитаем все четыре параметра:
	\[
	\begin{aligned}
		K(t) &= ( \\
			 &K_{ \alpha }(\widetilde{ \alpha_0} , \widetilde{ q_0} , \widetilde{ a_0}, \widetilde{ \beta_0}, t) \\ 
			 &K_{ q }(\widetilde{ \alpha_0} , \widetilde{ q_0} , \widetilde{ a_0}, \widetilde{ \beta_0}, t) \\ 
			 &K_{ a }(\widetilde{ \alpha_0} , \widetilde{ q_0} , \widetilde{ a_0}, \widetilde{ \beta_0}, t) \\ 
			 &K_{ \beta  }(\widetilde{ \alpha_0} , \widetilde{ q_0} , \widetilde{ a_0}, \widetilde{ \beta_0}, t) \\ 
			 &)
	\end{aligned}
	\]
	Теперь запишем совместную рекурсию для $ K_{ \alpha }, K_{ q}, K_{a}, K_{ \beta }$:
	\[
	\begin{aligned}
		\begin{cases}
			K_{ \alpha }(\widetilde{ \alpha _0} , \widetilde{ q_0} , \widetilde{ a_0} , \widetilde{ \beta _0} , 0) &= \widetilde{ \alpha _0} \\
			K_{ \alpha }(\widetilde{ \alpha _0} , \widetilde{ q_0} , \widetilde{ a_0} , \widetilde{ \beta _0} , t+1) &= \widetilde{ \alpha } ' \Bigl(  
				K_{ \alpha }( \ldots , t), K_{q} (\ldots , t), K_{a} (\ldots , t) , K_{ \beta }(\ldots , t)
			\Bigr)
		\end{cases}
	\end{aligned}
	\]
	Для остальных точно также.

\item[Результат] 
	Пусть начальное состояние $ q_0 a_0 \beta_0$ (стоим на самом левом символе), конечное  --- $ q_{z} a_{z} \beta _{z}$, причем $ z$ встречаем впервые. 
	То есть нам нужно вычислить функцию, которая переводит $ \widetilde{ a_0} + \widetilde{ b_0} m \longrightarrow a_z + \beta _z m$, если машина Тьюринга пришла сюда и не определена, если МТ зацикливается:
	\[
		t_z = \mu t[K_{q}(t) = z]
	.\] 
	Тогда результатом работы МТ будет
	\[
	\begin{aligned}
		\varphi (x) =& m \cdot  K _{\beta} \Bigl(0, 0, x \mod m, \lfloor \tfrac{x}{m} \rfloor , \\
					 &\mu t[K_{q}(0, 0, x \mod m, \lfloor \tfrac{x}{m} \rfloor, t) = z]\Bigr) +\\ 
				   &+ K_{a}(0, 0, x \mod m, \lfloor \tfrac{x}{m} \rfloor)
	\end{aligned}
	\]
			\end{description}
    \end{description} 
\end{proof}

