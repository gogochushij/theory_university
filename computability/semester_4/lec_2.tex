\begin{cor}
    Любую частично рекурсивную функцию можно представить так, что минимизация использовалась только один раз.
\end{cor}
\begin{proof}
    Сначала запишем для нее МТ, а потом постоим обратно функцию. В итоге получим эквивалентную функцию, причем по построению оператор минимизации использовался лишь один раз.
\end{proof}
\begin{cor}
	Функция, вычислимая за примитивно рекурсивное время \footnote{время, ограниченное примитивно рекурсивной функцией}, является примитивно рекурсивной.
\end{cor}
\begin{proof}
    В построении функции использовали минимизацию по числу шагов МТ,  поэтому, если работаем примитивно рекурсивное время, можем применить ограниченную минимизацию. 
\end{proof}

\subsection{Функция Аккермана}
Можно построить общерекурсивную функцию, которая растет быстрее любой примитивно рекурсивной. Из этого следует, что \prf\ не совпадает с \orf.
\begin{defn}[Функция Аккермана]\index{функция Аккермана}
	\selectedFont{Функция Аккермана}  --- функция от двух аргументов $ \alpha _n(x)$, которая определяется следующим образом:
	\[
		\begin{cases}
			a_0(x) &= x+1 \\ 
			a_{n+1}(x) &= a_{n}^{[x+2]}(x) = \underbrace{a_{n} (a_n( \ldots (x)))}_{x+2 \text{ раза}}  
		\end{cases}
	\] 
\end{defn}
\begin{thm}
	$ \alpha _{n}(n) \colon \N \to \N$  растет быстрее любой примитивно рекурсивной.
\end{thm}
\begin{proof*}
    <<Доказательство -- упражнение>>, занимает пару страниц, в ближайшее время появится здесь.
\end{proof*}

\chapter{Разрешимые и перечислимые множества}
\section{Определения}
\begin{defn}[Разрешимое множество]\index{разрешимое множество}
	Множество $ X \subseteq \N^{k}$ называется \selectedFont{разрешимым}, если его характеристическая функция вычислима\footnote{Это может быть частично рекурсивная функция, машина Тьюринга, $ \lambda$-функция...}.
\end{defn}
\begin{note}
    Любое конечное множество разрешимо. Пересечение, объединение, разность разрешимых тоже разрешимо.
\end{note}
\begin{thm}
	Множество $ X \subseteq \N$ разрешимо тогда и только тогда, когда $ X$ --- множество значений всюду определенной вычислимой неубывающей функции (или пустое множество).
\end{thm}
\begin{proof}
	~\begin{description}
		\item \boxed{ 1 \Longrightarrow 2} Можем в характеристической функции $ \chi_{X}(n)$ возвращать $ n$ вместо $ 1$, а в остальных значениях прошлое выданное. Эта функция подходит под описание.
        \item \boxed{ 2 \Longrightarrow 1} 
			Если множество конечно, то оно разрешимо, так как можем задать функцию $ \chi_X$ на конечном числе точек. 
			Если $ X$ бесконечно будем действовать, как описано далее.

			Пусть есть функция $ f$. Из нее хотим построить $ \chi_{X}$. Посчитаем $ \chi_{X}(n)$ так: 
			начнем с $ i = 0$
			\begin{itemize}
				\item вычислим $ f(i)$;
				\item если значение больше $ n$, то в следующих входах, значения будут еще больше, поэтому можем сразу вернуть  $ 0$;
				\item если меньше, то посчитаем $ f(i+1)$ и вернемся к предыдущему пункту;
				\item так как функция неубывающая и достигает всех значений из $ X$ (причем из бесконечно много, поэтому есть элемент больше $ n$), мы либо найдем значение больше $ n$ (тогда вернем $ 0$), либо равное (тогда вернем единицу).
			\end{itemize}
    \end{description} 
\end{proof}



