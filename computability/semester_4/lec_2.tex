\begin{cor}
    Любую частично рекурсивную функцию можно представить так, чтобы минимизация использовалась только один раз.
\end{cor}
\begin{proof}
    Сначала запишем для нее МТ, а потом постоим обратно функцию. В итоге получим эквивалентную функцию, причем по построению оператор минимизации использовался лишь один раз.
\end{proof}
\begin{cor}
	Функция, вычислимая за примитивно рекурсивное время \footnote{время, ограниченное примитивно рекурсивной функцией}, тоже является примитивно рекурсивной.
\end{cor}
\begin{proof}
    В построении функции использовали минимизацию по числу шагов МТ,  поэтому, если работаем примитивно рекурсивное время, можем применить ограниченную минимизацию. 
\end{proof}

\section{Функция Аккермана}
Можно построить общерекурсивную функцию, которая растет быстрее любой примитивно рекурсивной. Из этого следует, что \prf\ не совпадает с \orf.
\begin{defn}[Функция Аккермана]\index{функция Аккермана}
	\selectedFont{Функция Аккермана}  --- функция от двух аргументов $ \alpha _n(x)$, определенная следующим образом:
	\[
		\begin{cases}
			\alpha_0(x) &= x+1 \\ 
			\alpha_{n+1}(x) &= \alpha _{n}^{[x+2]}(x) = \underbrace{\alpha_{n} ( \alpha _n( \ldots (x)))}_{x+2 \text{ раза}}  
		\end{cases}
	\] 
\end{defn}
\begin{lm}\label{lm:akk_1}
	$ \alpha _n(x) \ge x + n+1$. В частности, $ \alpha _n(x) > x$.
\end{lm}
\begin{proof*}
    Докажем по индукции по $ n$.
	\begin{itemize}
		\item База: $ n = 0$. $  \alpha _0(x) = x+1 = x + 0 + 1$.
		\item Переход: $ n -1\to n $. 
			\begin{align*}
				\alpha _n(x) &= \alpha _{n-1} ^{[x+2]}(x) \ge  a_{n-1}^{[x+1]}(x) + 1 > \tag{т.к. $ \alpha _{n-1}(t) > t$ по предположению индукции} \\
							 & > \alpha _{n-1}^{[x]}(x) + 1 > \alpha _{n-1}^{[x-1]} + 1 > \ldots >\tag{продолжаем до кратности 1} \\
							 &> \alpha _{n-1} (x) + 1 \ge x + (n - 1) + 1 + 1 = x + n + 1
			\end{align*}
	\end{itemize}
\end{proof*}
\begin{lm}\label{lm:akk_2}
	Если $ x > y$, то $ \alpha _n (x) > \alpha _n(y)$.
\end{lm}
\begin{proof*}
    Индукция по $ n$.
	\begin{itemize}
		\item База: $ n = 0$. $  \alpha _0 (x) = x + 1 > y + 1 = \alpha _0(y)$.
		\item Переход: $ n-1 \to  n$. 
			\begin{align*}
				\alpha_n(x) &= \alpha _{n-1}^{[x+2]}(x) > \alpha _{n-1}^{[x+2]}(y) \tag{по предположению для $  n-1$} \\
							&> \alpha _{n-1}^{[x+1]}(y) > \ldots > \alpha _{n-1}^{[y+2]}(y) \tag{по \hyperref[lm:akk_1]{прошлой лемме \ref{lm:akk_1}}} \\
							&= \alpha _n(y)
			\end{align*}
	\end{itemize}
\end{proof*}
\begin{lm}\label{lm:akk_3}
	Если $ n > m$, то $  \alpha _n(x) > \alpha _m(x)$.
\end{lm}
\begin{proof*}
    Индукция по $ m$.
	\begin{itemize}
		\item База: $ m=0$. По \hyperref[lm:akk_1]{лемме \ref{lm:akk_1}} $ \alpha _{n}(x) \ge x + n + 1 \ge  x + 1 = \alpha _0(x)$. 
		\item Переход: $ m -1 \to  m$. По определению $  \alpha _n(x) = \alpha _{n-1}^{[x+2]}(x)$. Применим индукционное предположение ко всем $ \alpha _{n-1}$  и  заменим на $ \alpha _{m-1}$, после каждой замены значения будут уменьшаться:
			\[
				\alpha _{n-1}^{[x+2]}(x) > \alpha _{n-1}^{[x+1]}( \alpha _{m-1}(x)) > \ldots > \alpha _{n-1}\bigl( \alpha _{m-1}^{[x+1]}(x) \bigr) > \alpha _{m-1}^{[x+2]}(x)
			.\] 
	\end{itemize}
\end{proof*}
\begin{lm}\label{lm:akk_4}
	Если $ n > 1$, то $  \alpha _{n}(x) \ge \alpha _{n-1}^{[2]}(x)$.
\end{lm}
\begin{proof*}
    Рассмотрим 2 случая:
    \begin{itemize}
        \item Если $ x = 0$, то $ \alpha_{n}(x) = \alpha_{n - 1}^{[x + 2]}(x) = \alpha_{n - 1}^{[2]}(x)$.
        \item Иначе $ \alpha_{n}(x) = \alpha_{n - 1}^{[x + 2]}(x) > \alpha_{n - 1}^{[x + 1]}(x) \ge \alpha_{n - 1}^{[2]}(x)$
    \end{itemize}
\end{proof*}
\begin{lm}\label{lm:akk_5}
	Для любой \prf $ ~f(x_1, \ldots x_n)$ существует константа $ k$, что $ f(x_1, \ldots x_{n}) \le \alpha _k( \max \{x_1, \ldots, x_{n}\})$. Если $ f$ имеет 0 аргументов, подставим  $ 0$.
\end{lm}
\begin{proof*}
	\begin{itemize}
		\item Для примитивных функций подойдет $ k = 0$: для нульместного и одноместного нулей,  функцию $ s(x)$ и проекции $ I_{a}^{b}(x_1, \ldots , x_a)$ неравенство верно, так как $  \alpha _0(\max\overline{x}) = \max \overline{x} + 1$.
		\item Если применяется оператор суперпозиции для других функций с найденными $ k_{i}$, можно взять наибольшее из них (пусть $ k$), т.е. $h(\overline{x}),g_i(\overline{x}) \le \alpha _k(\max\overline{x}) \implies \max g_i(\overline{x}) \le \alpha _k(\max\overline{x})$. Докажем, что подойдет $ \alpha _{k+1}(x)$. 

			Пусть суперпозиция применяется к $ h(x_1, \ldots x_m)$ и $ g_{i}(x_1, \ldots x_{n})$.
			\begin{align*}
				h(g_1(\overline{x}), \ldots, g_m(\overline{x}))
				&\le 
				\alpha _{k} \bigl( \max_{i \in [1, m]} g_i(\overline{x})\bigr) 
				\tag{используем $\alpha_k$ для $h$} \\
				&\le
				\alpha _{k}( \alpha_k\bigl(\max \overline x) \bigr)
				\tag{используем $\alpha_k$ для $g_i$} \\
				&=
				\alpha_{k}^{[2]}(\max\overline{x}) 
				 \\
				&\le
				\alpha _{k+1}(\max\overline{x}) \tag{\hyperref[lm:akk_4]{лемма \ref{lm:akk_4}}}
			\end{align*}
		\item Если функция $ f(\overline{x}, n)$ получена из $ g(\overline{x}) $ и $ h(\overline{x}, n, t)$ с помощью примитивной рекурсии, сначала найдем $k$, чтобы $g(\overline{x}) \le \alpha _k( \max \overline{x})$ и $ h(\overline{x}, n, t) \le \alpha _k(\max(\overline{x}, n, t))$. 

			Оценим функцию $ f$.
			\[
			\begin{cases}
				f(\overline{x}, 0) = g(\overline{x})\\
				f(\overline{x}, n+1) = h(\overline{x}, n, f(\overline{x} , n))
			\end{cases}
			\] 
			Докажем, что $ f(\overline{x}, n) \le \alpha _k ^{[n+1]}( \max (\overline{x}, n))$.\footnote{Здесь под $ \max(\overline{x}, \text{что-то еще})$ подразумевается максимум по всем координатам и $ n$:  $ \max(x_1, \ldots , x_m, \text{что-то еще})$.}
			\begin{itemize}
				\item База: $ n = 0$. Верно, по определению  $ f$.
				\item Переход:  $ n \to n + 1$. 
					\begin{align*}
						f(\overline{x}, n+1)& 
											 \le \alpha _k\bigl( \max (\overline{x}, n, f(\overline{x}, n))\bigr) \le 
											 \alpha _k\Bigl( \max\bigl(\overline{x}, n, \alpha _k ^{[n+1]} (\max (\overline{x}, n))\bigr)\Bigr) 
											 \tag{индукционное предположение} \\
											& = \alpha _k\bigl( \alpha _{k}^{[n+1]} (\max (\overline{x}, n)) \bigr) = \alpha _{k}^{[n+2]} (\max (\overline{x}, n)) 
					\end{align*}
			\end{itemize}
			То есть 
			 \[
				 f(\overline{x}, n) \le \alpha _k^{[n+1]}(\max (\overline{x}, n)) < \alpha _k ^{[\max (\overline{x}, n) + 2]}(\max( \overline{x}, n)) = \alpha _{k+1}(\max (\overline{x}, n))
			 \] 
    \end{itemize}
\end{proof*}

\begin{lm}\label{lm:akk_6}
	Для любой \prf $ ~f(n)$ найдется такое  $ N \in \N$, что при $ n > N$ выполнено  $  \alpha _n(n) > f(n)$.
\end{lm}
\begin{proof*}
	По \hyperref[lm:akk_5]{лемме \ref{lm:akk_5}} для $ f(n) + 1$ найдется такое $ N$, что $  \alpha _N(n) \ge  f(n) + 1 > f(n)$. А тогда и для всех больших $ m \ge N$ верно неравенство $  \alpha _m(n) > f(n)$, в том числе и $\alpha _m(m) > f(m)$.
\end{proof*}
\begin{thm}
	$ \alpha _{n}(n) \colon \N \to \N$  растет быстрее любой примитивно рекурсивной.
\end{thm}
\begin{proof*}
	По \hyperref[lm:akk_5]{лемме \ref{lm:akk_6}} для любой примитивно рекурсивной функции есть константа $ N$, начиная с которой  $  \alpha _n(n) > f(n)$, то есть она будет расти быстрее.

	Из этого также следует, что $  \alpha _n(n)$ не \prf.
\end{proof*}
\begin{lm}\label{lm:akk_7}
	$ \alpha _n(x)$ является общерекурсивной функцией двух аргументов. В частности, $  \alpha _n(n)$ --- одного аргумента.
\end{lm}
\begin{proof*}
	Построим машину Тьюринга $ A$, вычисляющую $ \alpha _n(x)$ по $ n$ и  $ x$: $ x+2$ раза повторяем рекурсивно вызывать себя для $ n-1$ и результата рекурсивных вызовов. Когда доходим до $ n  = 0$, возвращаем число, увеличенное на один. 

	В итоге мы построили МТ строго по определению.
\end{proof*}


\chapter{Разрешимые и перечислимые множества}
\section{Определения}
\begin{defn}[Разрешимое множество]\index{разрешимое множество}
	Множество $ X \subseteq \N^{k}$ называется \selectedFont{разрешимым}, если его характеристическая функция вычислима\footnote{Это может быть частично рекурсивная функция, машина Тьюринга, $ \lambda$-функция...}.
\end{defn}
\begin{note}
    Любое конечное множество разрешимо. Пересечение, объединение, разность разрешимых тоже разрешимо.
\end{note}
\begin{thm}
	Множество $ X \subseteq \N$ разрешимо тогда и только тогда, когда $ X$ --- множество значений всюду определенной вычислимой неубывающей функции (или пустое множество).
\end{thm}
\begin{proof}
	~\begin{description}
		\item \boxed{ 1 \Longrightarrow 2} Можем в характеристической функции $ \chi_{X}(n)$ возвращать $ n$ вместо $ 1$, а в остальных значениях прошлое выданное. Эта функция подходит под описание.
        \item \boxed{ 2 \Longrightarrow 1} 
			Если множество конечно, то оно разрешимо, так как можем задать функцию $ \chi_X$ на конечном числе точек. 
			Если $ X$ бесконечно будем действовать, как описано далее.

			Пусть есть функция $ f$. Из нее хотим построить $ \chi_{X}$. Посчитаем $ \chi_{X}(n)$ так: 
			начнем с $ i = 0$
			\begin{itemize}
				\item вычислим $ f(i)$;
				\item если значение больше $ n$, то в следующих входах, значения будут еще больше, поэтому можем сразу вернуть  0;
				\item если меньше, то посчитаем $ f(i+1)$ и вернемся к предыдущему пункту;
				\item так как функция неубывающая и достигает всех значений из $ X$ (причем их бесконечно много, поэтому есть элемент больше $ n$), мы либо найдем значение больше $ n$ (тогда вернем $ 0$), либо равное (тогда вернем единицу).
			\end{itemize}
    \end{description} 
\end{proof}

