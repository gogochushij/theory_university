\lecture{6}{18 march}{\dag}
\section{Арифметическая иерархия}

\ref{свойство про проекцию (было)}
Можем считать $ A, B$ свойствами (предикатами):  $ A(x) \-> x \in A$

Тогда можем переписать так
\[
    A(x) \Longleftrightarrow \exists y \colon\quad B(x, y)
.\] 

Какие множества представимы в виде $ \forall y \colon B(x, y)$? Это равносильно
\[
    \neg \left( \exists  y ~ \neg B(x, y) \right) 
.\] 
Это поперечислимые (то есть дополнение до перечислимого)

\begin{defn}
    $ A \in \Sigma_{n} $, если его можно представить в виде 
    \[
	A(x) \Longleftrightarrow \exists y_1 \forall y_2 \exists y_3 \ldots B(x, y_1, \ldots , y_{n})
    ,\] 
    где $ B$ разрешимо.

     $ B \in \Pi_{n}$, если
     \[
	 A(x) \Longleftrightarrow \forall y_1 \exists y_2 \forall y_3 \ldots B(x, y_1, \ldots , y_{n})
     .\] 
\end{defn}
\begin{prop}
    \begin{enumerate}
        \item Определение не изменится, если разрешить несколько одинаковых вкванторов подряд, так как можем заменить повторные на один с помощью канторовой нумерации
	\item $ A(x) \in  \Sigma _{n} \Longleftrightarrow \neg (x) \in \Pi_{n}$
    \end{enumerate}
\end{prop}

\begin{thm}
    Если $ A(x), B(x) \in \Sigma _n $ (или $ \Pi_n$), то
     \[
	 A(x) \cap B(x) \in \Sigma_n , \quad B(x) \in \Sigma_n
    \] 
    (или $ \Pi_n$)
\end{thm}
\begin{proof}
    \begin{align*}
	A(x) & \Longleftrightarrow \exists y_1 \forall y_2 \exists y_3 \ldots P(x, y_1, \ldots y_{n}) \\
	B(x) & \Longleftrightarrow \exists z_1 \forall z_2 \exists z_3 \ldots Q(x, z_1, \ldots z_{n}) 
    \end{align*}
    \[
	A(x) \cap B(x) \Longleftrightarrow \underbrace{\exists y_1 \exists z_1}_{\exists c(y_1, z_1)} \forall y_2 \forall z_2 \ldots \underbrace{P(x, y_1, \ldots y_{n}) \cap Q(x, z_1, \ldots z_n)}_{l(c(y, z))}
    .\] 
\end{proof}

\begin{note}
    Аналогично можно определить $ \Sigma _n, \Pi_n$ для подмножеств $ \N^{m}$.
\end{note}
\begin{prop}
    %12:31
\end{prop}

\begin{thm}
    $ A \le _m B ,  ~ B \in \Sigma _n$, то $ A \in \Sigma _{n}$ (аналогично с $ \Pi_n$)
\end{thm}
\begin{proof}
    По определению
    % дописать
\end{proof}

\begin{thm}
    Для любого $ n >0$ в классе $ \Sigma_n $ (соответственно $ \Pi_n$ ) существует множество, универсальное для всех множеств в $ \Sigma _n$ (соответственно $ \Pi_n $ ) 
\end{thm}
\begin{note}
    Если $ A$ -универсальное в $ \Sigma _n$, то $ \overline{A}$ --- универсальное в $ \Pi_n$
\end{note}
\begin{proof}
    Для $ \Sigma _1$ -- перечислимое, уже строили.
    Для $ \Pi_2$
     \[
     \forall y \underbrace{\exists z \underbrace{R(x, y, z)}_{\text{разрешимо}}}_{P(x, y)} \Longrightarrow \forall y \underbrace{P(x, y)}_{\text{перечислимо}}
    .\] 
    Рассмотрим $ U(n, x, y)$ --- универсальное множество для перечислимых. Тогда 
     \[
	 T(n, x) = \forall y U(n, x, y) \text{ --- универсальное для } \Pi_2
    .\] 
    Следовательно,
    существует универсальное для $ \Sigma _2$ --- дополнение до $ T(n, x)$ 

    Продолжаем далее по индукции, для $ 3$ начинаем с  $ \Sigma _3$
    \[
	\Sigma _3 \colon \exists y \forall z \exists t R(x, y, z, t) \Longleftrightarrow \exists y \forall z P(x, y, z)
    .\] 
    Универсальное для $ \Sigma _3$::
    \[
	T(n, x) = \exists y \forall z U(n, x, y, z), \text{где } $ U$ \text{ --- универсальное перечислимое}
    .\] 
\end{proof}

% Дописать определения для множеств

\begin{thm}
    Универсальное множество для $ \Simga_n$ не принадлежит $ \Pi_n$ и наоборот
\end{thm}

Цель: доказать, что $ \forall \exists A \subset \N$, такое что $ \Sigma_n = \{X \mid X~ A \text{перечислимо}\}$
 
\section{Еще про $ T$}


