\section{Перечислимые множества}

\begin{defn}[]
	Подмножество $ \N^{k}$ называется \selectedFont{перечислимым}, если существует алгоритм, который выводит все его элементы в некотором порядке.
\end{defn}

\begin{thm}[Об эквивалентных определений]
	\begin{enumerate}
	    \item область определения вычислимой функции
		\item область значений вычислимой функции
		\item его полухарактеричтическая функция вычислима
		\item область значений  всюду определена вычислимой функцией
	\end{enumerate} 
\end{thm}
\begin{proof}
    $ $
    \begin{description}
        \item \boxed{ \text{перечислимость} \Longrightarrow 2} 
			Чтобы посчитать $ \chi_{X}(n)$, запускаем алгоритм, перечисляющий элементы множества, ждем $ n$.

			Если вывелось $ n$, то выводим $ \chi_{X}(n) = 1$, а иначе мы зациклились, то есть получили расходимость.
        \item \boxed{ 3 \Longrightarrow 1} 
			Очевидно --- для $ \chi_{X}$ 
        \item \boxed{ 1 \Longrightarrow \text{перечислимость}} 
			Пусть область определения вычисляется алгоритмом  $ B$.

			Будем запускать $ B$ по шагам:
			\begin{itemize}
				\item $ 1$ шаг на $ 0$
				\item $ 2$ шага на $ 0$, $ 2$ шага на $ 1$
				\item $ 3$ шага на $ 0, 1, 2$
				 \item и так далее
				 \item как только $ B$ закончил работу на некотором элементе, выводим этот элемент.
			\end{itemize}
			Этот алгоритм $ A$ перечисляет наше множество.
        \item \boxed{ 2 \Longrightarrow \text{перечислимость}} 
			Аналогично, но выводим значение функции на элементе, на котором мы останавливаемся.
        \item \boxed{ 1 \Longrightarrow 2} 
			Пусть $ X$ --- область определения функции, которая вычисляется алгоритмом $ A$.
			Рассмотрим следующую функцию:
			\[
				b(n) = \begin{cases}
					n, & \text{если } A \text{ заканчивает работать на } n \\
					\up , & \text{если } A \text{ зацикливается на } n
				\end{cases}
			\] 
			Теперь $ X$ --- область значений $ b(n)$, а она вычислима.
        \item \boxed{ \text{перечислимость} \Longrightarrow 4} 
			Пусть $ A$ --- алгоритм, перечисляющий $ X$. Рассмотрим любой $ n_0 \in X$.

			Построим функцию $ f$, которая всюду определена и $ X$ --- ее область значений.
			\[
				f(n)
				= \begin{cases}
					t, &\text{если на $ n$-ом шаге работы $ A$ появляется $ t$} \\
					n_0, & \text{если ничего не появляется}
				\end{cases}
			\] 
        \item \boxed{ 4 \Longrightarrow 2} Очевидно 
    \end{description} 
\end{proof}
\begin{note}
    Все области значений и определений не применимы к пустому множеству, которое тоже перечислимое.
\end{note}

% \begin{task}
%     % В определении можно повторять элементы. //
% \end{task}

\begin{thm}
    Объединение и пересечение перечислимых множеств тоже перечислимое.
\end{thm}

\begin{thm}[Пост]
    $ A$ разрешимо тогда и только тогда, когда $ A$ и $ \overline{A}$ перечислимые.
\end{thm}
\begin{proof}
    $ $
    \begin{description}
        \item \boxed{ 1 \Longrightarrow 2} 
			$ A $ разрешимо, можем рассмотреть $ \chi_{A}(n)$, которая вычислима.

			Построим полухрарактеристические для $ A$ и $ \overline{A}$:
			\begin{iteimze}
			\item если $ n \in A$, то $ \chi_{A}' = 1$, иначе $ \chi_{A}'(n)$ расходится
			\item для $ \overline{A}$ аналогично, только наоборот.
			\end{iteimze}
        \item \boxed{ 2 \Longrightarrow 1} 
			Запускаем одновременно по шагам алгоритм, перечисляющий $ A$ и $ \overline{A}$, ждем появления $ n$-ого. Рано или поздно должно появиться. Теперь, если его выдал алгоритм для $ A$, то $ \chi_A(n) = 1$, а если для $\overline{A}$, то $ \chi_A(n) = 0$.
    \end{description} 
\end{proof}


\begin{defn}[Проекция]
	Подмножество $ P \subseteq \N$ называется \selectedFont{проекцией} $ Q \subseteq \N^2$, если   \[
		x \in  P \Longleftrightarrow \exists y \colon  (x, y) \in  Q
	.\] 
\end{defn}
\begin{thm}[О проекции]
    $ P$ перечислимое тогда и только тогда, когда $ P$ --- проекция некоторого разрешимого $ Q$.
\end{thm}
\begin{proof}
    $ $
    \begin{description}
        \item \boxed{ 1 \Longrightarrow 2} 
			Пусть $ A$ --- алгоритм, перечисляющий $ P$. Тогда 
			\[
				Q \coloneqq \{(n, t) \mid n \text{ появляется в течение } t \text{ шагов работы } A\}
			.\] 
        \item \boxed{ 2 \Longrightarrow 1} 
			Проекция перечислимого перечислима: берем алгоритм, которые перечисляет $ Q$, но оставляем только первую координату.
    \end{description} 
\end{proof}

\begin{thm}[О графике]
    Частичная функция $ f\colon \N \to \N$ вычислима тогда и только тогда, когда ее график 
	\[
		F = \{(x, y) \mid f(x) \text{ определена и } $ f(x) = y$\}
	\] 
	перечислим.
\end{thm}
\begin{proof*}
    Упражнение
\end{proof*}


\begin{note}
	Различные обозначения в других источниках
	\begin{tabular}{c|c}
		перечислимое & разрешимое \\
		\hline
		вычислимо перечислимое & вычислимое \\
		полурекурсивное & рекурсивное \\
		полуразрешимое & decidable  \\
		enumerable & recursive \\
		semi-decidable & \ldots 
    \end{tabular}
\end{note}


\section{Универсальные функции}
Рассмотрим функцию $ U^{(m+1)}(n, \overline{x})$.

Обозначим за $ U_n(\overline{x})$ --- функция от $ m$ аргументов (получается из $ U$ фиксацией первого аргуменаа). Обозначается проекцией ???

\begin{defn}[универсальная функция]
	$ U(n, \overline{x})$ --- универсальная для класса $ K$ функция от $ m$ аргументов, если 
	\begin{itemize}
		\item $ \forall n \colon  U_n(\overline{x}) \in K$
		\item $ \forall f \in K ~ \exists n \colon f = U_{n}$
	\end{itemize}
\end{defn}
\begin{note}
    Универсальная функция существует только для счетных $ K$
\end{note}
\begin{note}
    Все рассматриваемые функции частичные
\end{note}


\begin{thm}
    Существует вычислимая функция $ 2^{x}$ аргументов, универсальная для класса вычислимых функций $ 1$-ого аргумента.
\end{thm}
\begin{proof}
	Запишем все коды МТ, вычисляющих функции $ 1$-ого аргумента, в порядке возрастания (сначала по длинне, затем в алфавитном).

	Пусть $ U(i, x)$ --- результат работы MT  $ M_i$ на входе $ x$. Сечение $ U_i$ соответствует МТ $ M_i$.
	% написать подробнее 12:20
\end{proof}

\begin{descriprion}
	$ \mathcal{F}^{m}$ --- вычислимые функции от $  m$ аргументов
	$ \mathcal{F}^{m}_{*}$ --- всюду определенные вычислимые функции от $  m$ аргументов
\end{descriprion}

\begin{cor}
    Существует $ U \in  \F^{m+1}$, универсальная для $ \F^{m}$.
\end{cor}
% \begin{note}
%     написать про канторову нумерацию для $ m$ и проекции $ c_i$
% \end{note}
\begin{proof}
	Используем канторовскую нумерацию $ m$-ок $ c(\overline{x})$.
	Универсальной для $ 2^{x}$ будет
	\[
		U'(n, \overline{x}) \coloneqq U(n, c(\overline{x}))
	.\] 
	Проверим это. Рассмотрим произвольную функцию $ f(\overline{x}) \in \F^{m}$.

	Определим 
	\[
		g(\overline{x})\coloneqq f(c_1(x), \ldots c_m(x))
	.\] 
	$ g$ вычислима, $ U$ --- универсальна, поэтому 
	\[
		\exists n \colon  U_n(x) = g(x)
	.\] 
	\[
		U'(n, \overline{x}) \stackrel{\text{по определению } U}{=} U(n, c(\overline{x})) \stackrel{\text{по определению } g}{=} g(c(\overline{x})) = \stackrel{\text{по определению } g}{=}  f(\overline{x}) = f(c_1(c(\overline{x})), \ldots c_{m}(c(\overline{x})))
	.\] 
	То есть $ U'$ действительно универсальная.
\end{proof}


\begin{thm}
    Не существует $ V \in \F_{*}^{2}$ универсальной для $ F^{*}$.
\end{thm}
\begin{proof}
    Предположим, что такая функция $ U \in  \F_{*}^{2}$ существует.

	Рассмотрим диагональную функцию:
	\[
		d'(n) = U(n, n) + 1 \in \F_{*}
	.\] 
	$d'(n)$ отличается от всех сечений $ U$: если $ \forall x \colon U(n, x) = d'(x)$, подставим $ x= n \colon U(n, n) \ne U(n, n ) + 1$. Противоречие. 
\end{proof}
\begin{note}
    Для класса частичных функций такое рассуждение не проходит, так как они могут быть не определены и прибавление единицы ничего не меняет для неопределенности.
\end{note}

\begin{thm}
	Существует вычислимая частичная функция, не имеющая всюду определенного вычислимого продолжения\footnote{то есть нельзя доопределить до всюду определенной вычислимой}.
\end{thm}
\begin{proof}
	Подходит функция $ d'(n) = U(n, n) + 1$, где $ U$ --- универсальная вычислимая функция\footnote{далее это обозначение по умолчанию определяет универсальную вычислимую частичную функицию $ U^{(m+1)}$ для вычислимых частичных функций $ m$ аргументов}.

	Пусть ее можно доопределить до вычислимой $ d''$:
	\begin{itemize}
		\item при $ n$, где $ d(n, n) $ определена, $ d''$ на один больше
		\item а где не определена, $ d''$ определена.
	\end{itemize}
	Поэтому $ d''$ отличается от всех сечений универсальной функции. Противоречие. 
\end{proof}


Аналогично можно ввести определения для перечислимых множеств
$ W \subset \N^2$
% ДОПИСАТЬ


\subsection{Перечислимое неразрешимое множество}
\begin{thm}
    Существует перечислимое неразрешимое множество.
\end{thm}
\begin{proof}
	Рассмотрим вычислимую $ f(x)$, не имеющую всюду определенного вычислимого продолжения. Ее область определения $ F$ --- перечислимое и неразрешимое.
	\begin{itemize}
		\item $ F$ перечислимое, так как оно является областью определения вычислимой функции
		\item $ F$ не разрешимое: предположим противное,
			рассмотрим 
			 \[
				 g(x) = \begin{cases}
					 f(x), & x \in F \text{ -- вычислимое, всюду определенное} &
					 0, & x \notin F \text{ -- продолжение } f
				 \end{cases}
			.\] 
	\end{itemize}
\begin{note}
	$ F = \{n \mid U(n, n) \text{ определено}\}$ --- переформулировка $ L_1$ (останавливающиеся на своем входе МТ.
\end{note}
\begin{note}
    $ \overline{F}$ --- пример неперечислимого множества.
\end{note}
\begin{note}
    Область определения универсальной функции --- перечислимое, но не разрешмое. 
	так как облатсь определения $ U(n, n)$ --- частично определенная не разрешимая.
\end{note}
\begin{note}
    <<Проблема остановки>> --- останавливается ли МТ $ M$ на входе $ x$ --- переформулировка принадлежности области определения универсальной функции.
\end{note}
\end{proof}


\begin{thm}
    Существует частичная вычислимая функция $ f\colon  \N \to  \{0, 1\}$, не имеющая всюду определенное вычислимое продолжение.
\end{thm}
\begin{proof}
    Определим
	\[
		d'''(n) = \begin{cases}
		1, & U(n, n) = 0 \\
		0, & U(n, n) > 0 \\
		\up, & U(n, n) \up
	\end{cases}
	\] 
	Любое доопределение будет отличаться от $ U(n, n)$
\end{proof}

\begin{cor}
    Существуют перечислимые $ X, Y$, такие что $ X  \cap Y = \varnothing$, не отделимые никаким разрешимым множеством. 
\end{cor}
\begin{proof}
	$ X = \{ n \mid d'''(n) = 0\}$ и  $ Y = \{n \mid d'''(n) =1 \}$
Существует $ C$ разделяющее.
% добавить картинку
\end{proof}

