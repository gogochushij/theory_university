\documentclass[11pt,a4paper]{article}
\usepackage[english, russian]{babel}
\usepackage{xltxtra}
\usepackage{polyglossia}
\usepackage{mathpazo}

\defaultfontfeatures{Ligatures=TeX,Mapping=tex-text}

\setmainfont{STIX2Text-Regular.otf}[
ExternalLocation={stix2font/},
BoldFont=STIX2Text-Bold.otf,
ItalicFont=STIX2Text-Italic.otf,
BoldItalicFont=STIX2Text-BoldItalic.otf
]
\setmathrm{STIX2Math.otf}[
ExternalLocation={stix2font/}
]


\usepackage{makeidx}
\usepackage{amssymb, amsthm, amsfonts}
\usepackage{amsmath}
\usepackage{mathtools}
\usepackage{needspace}
\usepackage{enumitem}
\usepackage{fdsymbol}
\usepackage{fontawesome}


% разметка страницы и колонтитул
\usepackage[left=1.5cm,right=2cm,top=1.5cm,bottom=1.1cm,bindingoffset=0cm]{geometry}
\usepackage{fancybox,fancyhdr}
\fancyhf{}
\fancyhead[R]{\thepage}
\fancyhead[L]{\rightmark}
\fancyfoot{}
\fancyhfoffset{0pt}
\addtolength{\headheight}{13pt}
\pagestyle{fancy}

% Отступы
\setlength{\parindent}{3ex}
\setlength{\parskip}{3pt}

\usepackage{graphicx}
\usepackage{hyperref}

\usepackage{import}
\usepackage{xifthen}
\usepackage{pdfpages}

\newcommand{\incfig}[1]{%
    \def\svgwidth{\columnwidth}
    \import{./figures/}{#1.pdf_tex}
}


\usepackage{xifthen}
\makeatother
\def\@lecture{}%
\newcommand{\lecture}[3]{
    \ifthenelse{\isempty{#3}}{%
        \def\@lecture{Лекция #1}%
    }{%
        \def\@lecture{Лекция #1: #3}%
    }%
    \subsection*{\@lecture}
    \marginpar{\small\textsf{\mbox{#2}}}
}
\makeatletter


\usepackage{xcolor}
\definecolor{Aquamarine}{cmyk}{50, 0, 17, 100}
\definecolor{ForestGreen}{cmyk}{76, 0, 76, 45}
\definecolor{Pink}{cmyk}{0, 100, 0, 0}
\definecolor{Cyan}{cmyk}{56, 0, 0, 100}
\definecolor{Gray}{gray}{0.3}


 % Цвета для гиперссылок
\definecolor{linkcolor}{HTML}{3f888f} % цвет ссылок
\definecolor{urlcolor}{HTML}{af0000} % цвет гиперссылок
 
\hypersetup{pdfstartview=FitH,  linkcolor=linkcolor,urlcolor=urlcolor, colorlinks=true}


\usepackage{mdframed}
\mdfsetup{skipabove=3pt,skipbelow=3pt}
\mdfdefinestyle{defstyle}{%
    linecolor=red,
	linewidth=1pt,leftline=true,topline=false,bottomline=false, rightline=false,%
    frametitlerule=false,%
    frametitlebackgroundcolor=red!0,%
    innertopmargin=0pt,innerbottommargin=4pt,innerleftmargin=7pt
    frametitlebelowskip=10pt,
    frametitleaboveskip=7pt,
}
\mdfdefinestyle{thmstyle}{%
    linecolor=cyan!100,
	linewidth=2pt,topline=false,bottomline=false,%
    frametitlerule=false,%
    frametitlebackgroundcolor=cyan!20,%
    innertopmargin=4pt,innerbottommargin=4pt,
    frametitlebelowskip=1pt,
    frametitleaboveskip=3pt,
}
\theoremstyle{definition}
\mdtheorem[style=defstyle]{defn}{Определение}

\newmdtheoremenv[nobreak=true,backgroundcolor=Aquamarine!10,linewidth=0pt,innertopmargin=0pt,innerbottommargin=7pt]{cor}{Следствие}
\newmdtheoremenv[nobreak=true,backgroundcolor=CarnationPink!20,linewidth=0pt,innertopmargin=0pt,innerbottommargin=7pt]{desc}{Описание}
\newmdtheoremenv[nobreak=true,backgroundcolor=Gray!10,linewidth=0pt,innertopmargin=0pt,innerbottommargin=7pt,font={\small}]{ex}{Пример}
\newmdtheoremenv[nobreak=false,backgroundcolor=cyan!10,linewidth=0pt,innertopmargin=0pt,innerbottommargin=7pt]{thm}{Теорема}
\newmdtheoremenv[nobreak=true,backgroundcolor=Pink!10,linewidth=0pt,innertopmargin=0pt,innerbottommargin=7pt]{lm}{Лемма}

\newtheorem*{st}{Утверждение}
\newtheorem*{prop}{Свойства}

\theoremstyle{plain}
\newtheorem*{name}{Обозначение}

\theoremstyle{remark}
\newtheorem*{rem}{Ремарка}
\newtheorem*{com}{Комментарий}
\newtheorem*{note}{Замечание}
\newtheorem*{prac}{Упражнение}
\newtheorem*{probl}{Задача}


\renewcommand{\proofname}{Доказательство}
\renewenvironment{proof}
{ \hspace{\stretch{1}}\\ \faSquareO\quad  }
{ \hspace{\stretch{1}}  \faSquare }

\newenvironment{proof*}
{ \hspace{\stretch{1}}\\ {\color{orange}\faSquareO}\quad  }
{ \hspace{\stretch{1}}  {\color{orange}\faSquare} }


\numberwithin{ex}{section}
\numberwithin{thm}{section}
\numberwithin{equation}{section}



\newcommand{\A}{\mathcal{A}}
\newcommand{\K}{\mathcal{K}}
\newcommand{\Z}{\mathbb{Z}}
\newcommand{\N}{\mathbb{N}}
\newcommand{\Real}{\mathbb{R}}
\newcommand{\Q}{\mathbb{Q}}
\newcommand{\Cm}{\mathbb{C}}
\newcommand{\Pm}{\mathbb{P}}
\newcommand{\E}{\mathbb{E}}

\renewcommand{\o}{o}
\renewcommand{\O}{\mathcal{O}}

\renewcommand{\le}{\leqslant}
\renewcommand{\ge}{\geqslant}


\newcommand{\ord}{\operatorname{ord}}
\newcommand{\lcm}{\operatorname{lcm}}
\newcommand{\sign}{\operatorname{sign}}
\newcommand{\sg}{\operatorname{sg}}
\newcommand{\bsg}{\operatorname{\overline{sg}}}

\newcommand{\Div}{\operatorname{Div}}
\newcommand{\Prime}{\operatorname{Prime}}
\newcommand{\expp}{\operatorname{ex}}
\newcommand{\ith}{\operatorname{ith}}


\def\mybf#1{\textbf{#1}}
\def\selectedFont#1{\textbf{#1}}
\def\prf{\textbf{ПРФ}}
\def\crf{\textbf{ЧРФ}}
\def\orf{\textbf{ОРФ}}
\def\prec{п.р.}
\def\orec{о.р.}

\newcommand{\const}{\textmd{const}}
\newcommand{\R}{\mathbf{R}}
\renewcommand{\S}{\mathbf{S}}
\newcommand{\M}{\mathbf{M}}
\newcommand{\F}{\mathcal{F}}



\title{Билеты к экзамену по функциональному программированию}
\author{Тамарин Вячеслав}

\makeindex
\begin{document}
\maketitle
\section{Сравнение функционального и императивного подходов к программированию}
\subsection{Императивное программирование}
Вычисление (программа) описывается в терминах \mf{инструкций}, изменяющих \mf{состояние} вычислителя.
\begin{itemize}
	\item Инструкции выполняются последовательно;
	\item Состояние изменяется инструкциями \mf{присваивания} значений изменяемым переменным;
	\item Если механизм условного исполнения (if, switch);
	\item Инструкции можно повторять с помощью циклов (while, for);
	\item Типы данных описываются с оглядкой на их физическое представление в памяти.
\end{itemize}
Такой стиль иногда называют стилем фон Неймана.
\begin{ccode}
long factorial(int n) {
	long res = 1;
	for (int i = 0; i < n; i++) 
		res *= i;
	return res;
}
\end{ccode}
Выполнение программы --- переход вычислителя из начального состояния в конечное с помощью последовательных инструкций.

Часть конечного состояния может интерпретироваться как результат вычислений.

\subsection{Функциональное программирование}
	Функциональная программа --- \mf{выражение}, ее выполнение --- вычисление (\mf{редукция}) этого выражения.
\begin{hscode}
factorial n = if n == 0 then 1 else n * factorial(n-1)
\end{hscode}
Выполнение программы --- редукция выражения с помощью \mf{подстановки} определений функций в места их ,,вызова'' с заменой формальных параметров на фактические.
\begin{minted}[fontsize=\footnotesize,numbersep=3pt,framesep=1mm,linenos,bgcolor=lbg]{hs}
factorial 3
-> if 3 == 0 then 1 else 3 * factorial (3 - 1)
-> ...
-> 3 * factorial (3 - 1)
-> 3 * if (3 - 1) == 0 then 1 else (3 - 1) * factorial ((3 - 1) - 1)
-> ...
-> 3 * 2 * 1 * 1 
-> 6
\end{minted}

\begin{minipage}{0.47\linewidth}
\begin{itemize}
	\item Нет состояний --- нет изменяемых переменных;
	\item Нет переменных --- нет присваивания;
	\item Нет циклов, так как нет различий между итерациями -- состояниями
	\item Последовательность не важна, поскольку выражения независимы.
\end{itemize}
\end{minipage}
\hfill
\begin{minipage}{0.47\linewidth}
	\begin{itemize}
		\item Рекурсия --- замена циклов;
		\item Функции высших порядков;
		\item Сопоставление с образцом;
		\item Все функции --- чистые.
    \end{itemize}
\end{minipage}
\subsection{Сильные стороны ФП}
\begin{itemize}
	\item Регулярный синтаксис, удобство анализа кода;
	\item Мощная типизация, при этом можно практически не использовать типы в коде за счет эффективным алгоритмам вывода типов;
	\item Возможность генерации программ по набору свойств;
	\item Эффективная доказуемость свойств программ алгебраическими методами;
	\item Высокоурвневые оптимизации на базе эквивалентных преобразований.
\end{itemize}

\section{Основы $ \lambda$-исчисления. $ \lambda$-термы, свободные и связанные переменные.}
$ \lambda$-исчисление --- формальная система, лежащая в основе ФП. Разработано Алонзо Чёрчем в 1930-х для формализации и анализа понятия вычислимости.

\noindent
В \LMCисчислении тремя основными понятиями являются:
\begin{itemize}
	\item применение (аппликация, application) --- задает синтаксис применения функции к ее фактическим аргументам.
		Применение функции $ F$ к аргументу $ X$ записывается как $ FX$;
	\item лямбда-абстрацкия (abstraction) --- описывает синтаксис определения функции на основе параметризованного выражения, представляющего ее тело. Абстракция по x: $ \lam{x}{F}$;
	\item редукция (reduction) --- определяет отношение вычисления, основывающегося на подстановке фактических параметров вместо формальных.
\end{itemize}
\begin{defn}[$ \lambda$-термы]
	Множество \mf{$ \lambda $-термов} $ \Lambda$ индуктивно строится из переменных  $ V = \{x, y, z, \ldots \}$ с помощью применения и абстракции:
\[
\begin{aligned}
	x \in V && \Longrightarrow &&& x \in \Lambda \\
	M, N \in \Lambda && \Longrightarrow &&& (MN) \in \Lambda \\
	M \in \Lambda, ~  x \in  V && \Longrightarrow &&& (\lam{x}{M}) \in \Lambda
\end{aligned}
\]
В абстрактном синтаксисе 
\[
	\Lambda \Coloneqq V \mid (\Lambda \Lambda) \mid (\lam{V}{\Lambda})
.\] 
\end{defn}
Терм может быть \textit{переменной} (например, $ x \in \Lambda$), \textit{аппликацией} (например, $ (xy) \in \Lambda$) или \textit{лямбда-абстракцией} (например, $ \lam{x}{((x(yx)))} \in  \Lambda$).

\begin{defn}[Подтермы]
	Множество \mf{подтермов} терма $ Q$ определяется индуктивно:
	\[
	\begin{aligned}
		&\subterms(x) && = && \{x\} \\
		&\subterms(MN) && = && \{MN\} \cup \subterms(M) \cup \subterms(N) \\
		&\subterms(\lam{x}{M}) && = && \{\lam {x}{M}\} \cup \subterms(M)
	\end{aligned}
	\]
\end{defn}


\subsection{Соглашения}
Общеприняты следующие соглашения для термов:
\begin{itemize}
	\item Внешние скобки опускаются
	\item Применение  ассоциативно \textit{влево}:
		\[
			FXYZ \equiv (((FX)Y)Z)
		.\] 
	\item Абстракция ассоциативна \textit{вправо}:
		\[
			\lam{x y z}{M} \equiv (\lam{x}{(\lam{y}{(\lam{z}{M})})})
		.\] 
	\item Тело абстракции простирается вправо насколько это возможно:
		\[
			\lam{x}{MNK} \equiv \lam{x}{(MNK)}
		.\] 
\end{itemize}

\begin{defn}[Свободные переменные]
	Множество \mf{свободных переменных} в терме $ Q$ ($ FV(Q)$) определяется следующим образом:
	\[
	\begin{aligned}
		& FV(x) && = && \{x\} \\
		& FV(MN) && = && FV(M) \cup FV(N) \\
		& FV(\lam{x}{M} && = && FV(M) \setminus \{x\}
	\end{aligned}
	\]
\end{defn}

\begin{defn}[Связвнные переменные]
	Множество \mf{связанных переменных} в терме $ Q$ ($ BV(Q)$) определяется следующим образом:
	\[
	\begin{aligned}
		& BV(x) && = && \varnothing \\
		& BV(MN) && = && BV(M) \cup BV(N) \\
		& BV(\lam{x}{M} && = && BV(M) \cup  \{x\}
	\end{aligned}
	\]
\end{defn}

\subsection{Комбинаторы}
\begin{defn}[Комбинатор]
$ M$ --- \mf{замкнутый $ \lambda$-терм (комбинатор)}, если $ FV(M) = \varnothing$. 

\noindent
Множество замкнутых $ \lambda$-термов обозначается $ \Lambda^{0}$.
\end{defn}
\begin{enumerate}[noitemsep]
	\item $ I = \lam{x}{x}$
	\item $ \omega = \lam{x} {x x}$
	\item $ \Omega = \omega \omega = \left( \lam{x}{x x} \right) \left( \lam{x}{x x} \right) $
	\item $ K = \lam{x y} {x}$
	\item  $ K_{*} = \lam{x y}{y}$
	\item $ C = \lam{f x y}{f y x}$ 
	\item $ B = \lam{f g x}{f (g x)}$
	\item $ S = \lam{f g x}{f x (g x)}$
\end{enumerate} 


\section{Подстановка $ \lambda$-терма. Лемма подстановки.}
\begin{defn}[Подстановка]
	\mf{Подстановка} терма $ N$ вместо \textit{свободных} вхождений переменной $ x$ в терм $ M$ ($ \subst{x}{N}{M}$) задается следующими правилами:
	\[
	\begin{aligned}
		& \subst{x}{N}{x} && = && N \\
		& \subst{x}{N}{y} && = && y \\
		& \subst{x}{N}{PQ} && = && \left( \subst{x}{N}{P} \right) \left( \subst{x}{N}{Q} \right)   \\
		& \subst{x}{N}{\lam{x}{P}} && = && \lam{x}{P} \\
		& \subst{x}{N}{\lam{y}{P}} && = && \lam{y}{\subst{x}{N}{P}}, && \text{ если } y \notin FV(N)\\
		& \subst{x}{N}{\lam{y}{P}} && = && \lam{x}{\subst{y}{z}{P}}\left( \subst{y}{z}{P} \right) , && \text{ если } y \in FV(N)
	\end{aligned}
	\]
	Предполагается, что $ x$ и $ y$ различны, а $ z$ --- ,,свежая'' переменная, то есть $ z \notin FV(P) \cup FV(N)$.
\end{defn}
\begin{ex}[Пример подстановки]
    \[
	\subst{x}{uv}{\left( \left( \lam{\color{red}x}{(\lam{\color{green}x}{{\color{green}x}z}){\color{red}x}} \right) x  \right) } = 
	\left( \lam{\color{red}x}{(\lam{\color{green}x}{{\color{green}x}z})\color{red}x} \right) (uv)
    .\] 
\end{ex}
\begin{lm}[подстановки]
	Пусть $ M, N, L \in \Lambda$, $ x \not\equiv y $ и $ x \notin FV(L)$. Тогда 
	\[
		\subst{y}{L}{\subst{x}{N}{M}} = \subst{x}{\subst{y}{L}{N}}{\subst{y}{L}{M}}
	.\] 
\end{lm}

\section{$ \alpha $- и $ \beta $-конверсии. $ \eta$-конверсия и экстенсиональная эквивалентность.}
\subsection{$ \alpha $-эквивалентность}
Позволяет менять переменную на ,,свежую''.
Главная аксиома $ \alpha $-преобразования:
\[
	\tag{правило $\alpha$}
	\lam{x}{M} \eqa \lam{y}{\subst{x}{y}{M}}, \text{ если } y \notin FV(M)
.\] 
Аналогично $ \beta $-эквивалентности можно определить совместимость.
\subsection{$ \beta $-эквивалентность}
\subsubsection{$ \beta $-редукция}
Одношаговая $ \beta $-редукция для редекса $ (\lam{x}{M}) N$:
 \[
	 (\lam{x}{M})N \longrightarrow_{\beta } \subst{x}{N}{M}
.\] 
Сокращение $ \beta $-редекса --- результат подстановки в правой части:
\[
	I z = (\lam{x}{x}) z \longrightarrow_{\beta } \subst{x}{z}{x} = z
.\] 

Дополнение определения одношаговой $ \beta $-редукции:
\[
\begin{aligned}
	& M \lba N && \Rightarrow && ZM \lba ZN \\
	& M \lba N && \Rightarrow && MZ \lba NZ \\
	& M \lba N && \Rightarrow && \lam{x}{M} \lba \lam{x}{N}
\end{aligned}
\]
\subsubsection{Многошаговая $ \beta $-редукция}
Это рефлексивное транзитивное замыкание одношаговой:
\[
	\tag{правила совместимости}
\begin{aligned}
	& M \lba N && \Rightarrow && M \llba N \\
	& && && M \llba M \\
	& M \llba N, ~ N \llba L && \Rightarrow && M \llba L
\end{aligned}
\]
\begin{defn}[$ \beta $-эквивалентность]
	Отношение \mf{$ \beta $-эквивалентности ($ \beta $-конвертируемости)} (нотация $ \eqb $) --- симметричное транзитивное замыкание многошаговой редукции:
	\[
	\begin{aligned}
		& M \llba N && \Rightarrow && M \eqb N \\
		& M \eqb N && \Rightarrow  && N \eqb M \\
		& M \eqb N, ~ N \eqb L && \Rightarrow  && M \eqb L
	\end{aligned}
	\]
\end{defn}

\mf{Основная схема аксиом}: для любых $ M, N \in \Lambda$:
\[
	(\lam{x}{M})N \eqb \subst{x}{N}{M}
.\] 
Если $ M \eqb N$ доказуемо в $ \lambda$-исчислении, пишут 
 \[
\lambda \vdash M \eqb N
.\] 
\subsection{$ \eta$-эквивалентность}
Схема аксиом $ \eta$-преобразования:
\[
	\tag{правило $ \eta$ }
	\lam{x}{Mx} \eqn M, \text{ если } x \notin FV(M)
.\] 
Аналогично можем определить правила и совместимость, как для $ \beta $.

Смысл этой эквивалентности в том, что поведение данных двух термов одинаково; для произвольного $ N$ будет верно
\[
	(\lam{x}{M})N \eqb MN
.\] 

\end{document}
