\textbf{Контакты:} \url{sokolov.dmt@gmail.com}, 

Видимо будет письменный экзамен.

Есть прошлогодний конспект, там есть существенные ошибки, плюс курс немного отличается.

\chapter{Введение}
\section{Информация по Хартли}
Пусть у нас есть конечное множество объектов $ A$. Выдернем какой-то элемент.

Мы хотим придумать описание этого элемента, которое будет отличать его от всех остальных.

Самый простой вариант --- число битов требуемое для записи объекта.

Свойства, которые мы хотим получить от меры $ \chi(A)$:
\begin{enumerate}
    \item $ \chi$ дает нам оценку на длину описаний
	\item $ \chi(A \cap B) \le \chi(A) + \chi(B)$
	\item 
Если наше множество $ A \coloneqq  B \times C$, то можно описать для $ B$ и для  $ C$, поэтому можно ограничить:
\[
	\chi(A)  \le \chi(B) + \chi(C)
.\] 
\end{enumerate} 

\begin{defn}[Информация по Хартли]\index{информация по Хартли}
	$ \chi(A) \coloneqq  \log \lvert A \rvert $
\end{defn}
\begin{note}
Очевидно, второе свойство выполнено для такого определения. В третьем даже равенство. 

Описание --- например, битовая строка. Если логарифм нецелый, округляем вверх.
\end{note}
Пусть $ A  \subset X \times Y$. Обозначим проекции $ A_X$ и $ A_{Y}$.
\begin{figure}[ht]
    \centering
    \incfig{axy-img}
    \label{fig:axy-img}
\end{figure}
Здесь 
\begin{enumerate}
	\item $ \chi(A) \ge 0$
	\item $ \chi(A_{X}) \le \chi(A)$
	\item $ \chi(A) \le \chi(A_{X}) + \chi(A_{Y})$
\end{enumerate} 

Рассмотрим такой пример: 
\begin{figure}[ht]
    \centering
    \incfig{diag-img}
    \label{fig:diag-img}
\end{figure}
здесь, зная первую координату, можно сразу понять вторую.

Попробуем усилить третье свойство:
\begin{enumerate}
	\item[3'.] $ \chi(A) \le \chi(A_{X}) + \chi_{Y \mid X} (A)$, где $ \chi _{Y \mid X}(A)$ --- описание $ Y$ при условии $ X$.
\end{enumerate} 
Как будем определять $ \chi_{Y \mid X} (A)$?
Можно взять $ \max_{x \in X} \log(A(x))$.

Теперь для диагонального множество $ \chi_{Y \mid X}$ просто обнуляется и неравенство переходит в равенство.

Но если взять такие множества. 
\begin{figure}[ht]
    \centering
    \incfig{corner-img}
    \label{fig:corner-img}
\end{figure}
Во-первых, на первой картинке передав $ x$ столбца придется передавать  и $ y$ тоже. Во-вторых, мы не сможем отличить эти множества.
\begin{prac}
	Пусть $ A \subset X \times Y \times Z$.
	Доказать
	\[
		2 \chi(A) \le \chi(A_{XY}) + \chi(A_{XZ}) + \chi(A_{YZ})
	.\] 
\end{prac}
\begin{proof}
    Неравенство из условия равносильно тому, что 
    $$ |A|^2 \leq |A_{12}|\cdot |A_{23}|\cdot|A_{13}|, $$ 
    так как $\log$ --- возрастающая функция. Пусть 
    $$ A_{12} = \{(x_1,y_1),(x_2,y_2),\\ \ldots,(x_n,y_n)\}. $$ 
    Тогда $ A $ 
    состоит из множеств 
    %
    $$ B_i = \{(x_i,y_i,a_{i1}),\ldots,(x_i,y_i,a_{ij_i})\}, $$ 
    %
    и $|A| = j_1 + j_2 + \dots + j_n$. Рассмотрим теперь множество $A_{13}\times A_{23}$, состоящее из точек $(x_i,a_j,y_s,a_k)$. Тогда понятно, что все точки вида $(x_i,a_{ik},y_i,a_{is}), $ где $ k,s \leq j_i $, лежат в нашем множестве, а таких точек $j_1^2+j_2^2+\dots+j_n^2$. Значит, 
    %
    $$ |A_{13}|\cdot|A_{23}| \geq |A_{23}\times A_{13}| \geq j_1^2+j_2^2+\ldots+j_n^2. $$ 
    %
    Таким образом, $|A_{12}|\cdot |A_{23}|\cdot|A_{13}| \geq n(j_1^2+j_2^2+\dots+j_n^2) \geq (j_1 + j_2 + \dots + j_n)^2 = |A|^2$.
    %
    Здесь используется неравенство о среднем арифметическом и среднем квадратичным.
\end{proof}

\subsection{Применение информации}
Обозначим $ [n] \coloneqq \{1, \ldots , n\}$.
Первый игрок выбирает одно число, а второй должен угадывать.

Если два варианта игры:
\begin{itemize}
	\item Адаптивная --- ответ сразу
	\item Сначала пишем все запросы, а потом получаем все ответы.
\end{itemize}
Очевидно, что нам потребуется не менее логарифма запросов: нарисуем дерево, где вершины -- запросы, по двум ребрам можно перейти в зависимости от ответа. Листья должны содержать $ [n]$, поэтому глубина дерева не менее логарифма.
\begin{figure}[ht]
    \centering
    \incfig{graph-img}
    \label{fig:graph-img}
\end{figure}

Теперь подумаем с точки зрения теории информации. Пусть $ B \coloneqq Q_1 \times \ldots \times Q_h$, $ h$ --- число запросов, $ Q_i$ --- ответ на запрос по некоторому протоколу.
Хотим минимизировать $ h$.

Рассмотрим $ ([n], B)$ --- все возможные пары ---. Нас интересует множество  $ A \subseteq ([n], B)$ --- соответствует некоторым корректным запросам, здесь записаны ответы нашего протокола.
\[
	A = \{(m, b) \mid b = (q_1, \ldots , q_h), ~ m \text{ --- согласовано с ответом}\}
.\] 

\begin{enumerate}
	\item $ \chi _{[n] \mid B}(A) = 0$. Ответы на запросы должны однозначно определять число $ m$. Это свойство говорит о корректности протокола, то есть нам ничего не нужно, чтобы, зная ответы, получить $ m$.
	\item $ \log n \le \chi(A)$, так как каждому числу от $1$ до $n$ соответсвует правильный ответ со стороны протокола. С другой стороны, $ \chi(A) \le \chi_{B}(A) + \chi_{[n] \mid B}(A) = \chi_B(A) \le \chi\left( B \right) \le \sum_{i=1}^{h} \chi(Q_i) = h $. 
		Итого
		\[
		\log n \le  h
		.\] 
\end{enumerate} 

\subsubsection{Другая формулировка}
Пусть теперь за ответ <<да>> мы платим 1, а за <<нет>> 2. И мы хотим минимизировать не число запросов, а стоимость в худшем случае.
 \[
Q_i \stackrel{?}{\in} T_i
.\] 
Здесь на $i$-ом шаге узнаем, принадлежит ли наше число множеству $T_i$. Пусть $ A_i$ --- множество возможных $ x$ (ответов) перед шагом $ i$. В начале это все $ [n]$, в конце -- одно число.
\[
	A_i = \{a \in [n] \mid a \text{ согласовано с } Q_1, \ldots Q_{i-1}\}
.\] 
Стратегия минимальной цены бита информации: берем такое $ T_i$, что
 \[
	 2\bigl(\chi(A_i) - \chi(\underbrace{T_i}_{A_{i+1}})\bigr) = \chi(A_i) - \chi(\underbrace{A \setminus T_i}_{A_{i+1}})
.\] 
Докажем, что эта стратегия оптимальна. То есть для любой другой стратегии найдется число, с которым мы заплатим больше.

Если заплатили 1, то перешли в  $ A_i \to T_i$. Если заплатили 2, то $ A_i \to A_{i} \setminus T_i$. Заметим, что каждый раз мы заплатили за каждый бит одинаково.

Докажем оптимальность. Пусть второй игрок меняет число, чтобы мы заплатили как можно больше, причем он знает нашу стратегию.

Если в нашем неравенстве знак $ \ge $, он будет направлять на по <<нет>>, а при  $ \le$ <<да>>, за счет чего каждый бит он будет отдавать по цене большей, чем, если бы мы действовали в точности по стратегии. 

Следовательно, любая другая стратегия будет требовать большего вклада.

Можем решить уравнение на $ T_i$, должно получиться:
 \[
	 \Phi(\lvert T_i \rvert ) = \lvert A_i \rvert, \quad \Phi \text{ --- золотое сечение} 
.\] 

\begin{prac}[Задача про взвашивания монеток]
	Есть $ n$ монеток и рычажные весы. Хотим найти фальшивую (она одна). 
\begin{enumerate}
    \item Пусть $ n = 30$ и весы показывают, что больше, что меньше. Теперь запрос приносит $  \log 3$ информации, так как три ответа.
		\[
			\log 30 \le  \sum_{i=1}^{h} \chi(a_i) \le h \log 3 
		.\] 
	\item $ n = 15$, но мы не знаем относительный вес фальшивой монеты. В прошлом неравенстве можно заменить  $ 30$ на $ 29$. Если в какой-то момент у нас было неравенство, можем в конце узнать не только номер, но и относительный вес, поэтому у нас $ 29$ исходов.
	\item Вопрос: можно ли при $ n=14$?  Нет. 
\end{enumerate} 
\end{prac}
