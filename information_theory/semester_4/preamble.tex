\usepackage[english, russian]{babel}
\usepackage{xltxtra}
\usepackage{polyglossia}
\usepackage{mathpazo}

\defaultfontfeatures{Ligatures=TeX,Mapping=tex-text}

\setmainfont{STIX2Text-Regular.otf}[
ExternalLocation={stix2font/},
BoldFont=STIX2Text-Bold.otf,
ItalicFont=STIX2Text-Italic.otf,
BoldItalicFont=STIX2Text-BoldItalic.otf
]
\setmathrm{STIX2Math.otf}[
ExternalLocation={stix2font/}
]


\usepackage{makeidx}
\usepackage{amssymb, amsthm, amsfonts}
\usepackage{amsmath}
\usepackage{mathtools}
\usepackage{needspace}
\usepackage{enumitem}
\usepackage{fdsymbol}
\usepackage{fontawesome}


% разметка страницы и колонтитул
\usepackage[left=1.5cm,right=2cm,top=1.5cm,bottom=1.1cm,bindingoffset=0cm]{geometry}
\usepackage{fancybox,fancyhdr}
\fancyhf{}
\fancyhead[R]{\thepage}
\fancyhead[L]{\rightmark}
\fancyfoot{}
\fancyhfoffset{0pt}
\addtolength{\headheight}{13pt}
\pagestyle{fancy}

% Отступы
\setlength{\parindent}{3ex}
\setlength{\parskip}{3pt}

\usepackage{graphicx}
\usepackage{hyperref}

\usepackage{import}
\usepackage{xifthen}
\usepackage{pdfpages}

\newcommand{\incfig}[1]{%
    \def\svgwidth{\columnwidth}
    \import{./figures/}{#1.pdf_tex}
}


\usepackage{xifthen}
\makeatother
\def\@lecture{}%
\newcommand{\lecture}[3]{
    \ifthenelse{\isempty{#3}}{%
        \def\@lecture{Лекция #1}%
    }{%
        \def\@lecture{Лекция #1: #3}%
    }%
    \subsection*{\@lecture}
    \marginpar{\small\textsf{\mbox{#2}}}
}
\makeatletter


\usepackage{xcolor}
\definecolor{Aquamarine}{cmyk}{50, 0, 17, 100}
\definecolor{ForestGreen}{cmyk}{76, 0, 76, 45}
\definecolor{CarnationPink}{cmyk}{0, 0.349, 0.2118, 0}
\definecolor{Pink}{cmyk}{0, 100, 0, 0}
\definecolor{Cyan}{cmyk}{56, 0, 0, 100}
\definecolor{Gray}{gray}{0.3}


 % Цвета для гиперссылок
\definecolor{linkcolor}{HTML}{3f888f} % цвет ссылок
\definecolor{urlcolor}{HTML}{af0000} % цвет гиперссылок
 
\hypersetup{pdfstartview=FitH,  linkcolor=linkcolor,urlcolor=urlcolor, colorlinks=true}


\usepackage{mdframed}
\mdfsetup{skipabove=3pt,skipbelow=3pt}
\mdfdefinestyle{defstyle}{%
    linecolor=red,
	linewidth=1pt,leftline=true,topline=false,bottomline=false, rightline=false,%
    frametitlerule=false,%
    frametitlebackgroundcolor=red!0,%
    innertopmargin=0pt,innerbottommargin=4pt,innerleftmargin=7pt
    frametitlebelowskip=10pt,
    frametitleaboveskip=7pt,
}
\theoremstyle{definition}
\mdtheorem[style=defstyle]{defn}{Определение}

\newmdtheoremenv[nobreak=true,backgroundcolor=Aquamarine!10,linewidth=0pt,innertopmargin=0pt,innerbottommargin=7pt]{cor}{Следствие}
\newmdtheoremenv[nobreak=true,backgroundcolor=CarnationPink!30,linewidth=0pt,innertopmargin=0pt,innerbottommargin=7pt]{desc}{Описание}
\newmdtheoremenv[nobreak=true,backgroundcolor=Gray!10,linewidth=0pt,innertopmargin=0pt,innerbottommargin=7pt,font={\small}]{ex}{Пример}
\newmdtheoremenv[nobreak=false,backgroundcolor=cyan!10,linewidth=0pt,innertopmargin=0pt,innerbottommargin=7pt]{thm}{Теорема}
\newmdtheoremenv[nobreak=true,backgroundcolor=Pink!10,linewidth=0pt,innertopmargin=0pt,innerbottommargin=7pt]{lm}{Лемма}

\newtheorem*{st}{Утверждение}
\newtheorem*{prop}{Свойства}

\theoremstyle{plain}
\newtheorem*{name}{Обозначение}

\theoremstyle{remark}
\newtheorem*{rem}{Ремарка}
\newtheorem*{com}{Комментарий}
\newtheorem*{note}{Замечание}
\newtheorem*{prac}{Упражнение}
\newtheorem*{probl}{Задача}


\renewcommand{\proofname}{Доказательство}
\renewenvironment{proof}
{ \hspace{\stretch{1}}\\ \faSquareO\quad  }
{ \hspace{\stretch{1}}  \faSquare }

\newenvironment{proof*}
{ \hspace{\stretch{1}}\\ {\color{orange}\faSquareO}\quad  }
{ \hspace{\stretch{1}}  {\color{orange}\faSquare} }


\numberwithin{ex}{section}
\numberwithin{thm}{section}
\numberwithin{equation}{section}



\newcommand{\A}{\mathcal{A}}
\newcommand{\K}{\mathcal{K}}
\newcommand{\Z}{\mathbb{Z}}
\newcommand{\N}{\mathbb{N}}
\newcommand{\Real}{\mathbb{R}}
\newcommand{\Q}{\mathbb{Q}}
\newcommand{\Cm}{\mathbb{C}}
\newcommand{\Pm}{\mathbb{P}}
\newcommand{\E}{\mathbb{E}}

\renewcommand{\o}{o}
\renewcommand{\O}{\mathcal{O}}

\renewcommand{\le}{\leqslant}
\renewcommand{\ge}{\geqslant}


\newcommand{\ord}{\operatorname{ord}}
\newcommand{\lcm}{\operatorname{lcm}}
\newcommand{\sign}{\operatorname{sign}}
\newcommand{\sg}{\operatorname{sg}}
\newcommand{\bsg}{\operatorname{\overline{sg}}}

\newcommand{\Div}{\operatorname{Div}}
\newcommand{\Prime}{\operatorname{Prime}}
\newcommand{\expp}{\operatorname{ex}}
\newcommand{\ith}{\operatorname{ith}}


\def\mybf#1{\textbf{#1}}
\def\selectedFont#1{\textbf{#1}}
\def\prf{\textbf{ПРФ}}
\def\crf{\textbf{ЧРФ}}
\def\orf{\textbf{ОРФ}}
\def\prec{п.р.}
\def\orec{о.р.}

\newcommand{\const}{\textmd{const}}
\newcommand{\R}{\mathbf{R}}
\renewcommand{\S}{\mathbf{S}}
\newcommand{\M}{\mathbf{M}}
\newcommand{\F}{\mathcal{F}}


\usepackage{tikz}
\usetikzlibrary{shapes}

\tikzset{
  dot hidden/.style={},
  line hidden/.style={},
  dot colour/.style={dot hidden/.append style={color=#1}},
  dot colour/.default=black,
  line colour/.style={line hidden/.append style={color=#1}},
  line colour/.default=black
}

\usepackage{xparse}

\NewDocumentCommand{\domino}{mm}{
\begin{tikzpicture}[x=2em,y=2em,radius=0.1]
\draw[rounded corners=0.5,line hidden] (0,0) rectangle (1,2);

\draw[line hidden] (0,1) -- (1,1);
\ifodd#1
  \fill[dot hidden] (0.5,1.5) circle;
\fi
\ifnum#1>1
   \fill[dot hidden] (0.2,1.2) circle;
   \fill[dot hidden] (0.8,1.8) circle;
   \ifnum#1>3
      \fill[dot hidden] (0.8,1.2) circle;
      \fill[dot hidden] (0.2,1.8) circle;
   \fi
   \ifnum#1>5
      \fill[dot hidden] (0.2,1.5) circle;
      \fill[dot hidden] (0.8,1.5) circle;
   \fi
\fi
\ifodd#2
   \fill[dot hidden] (0.5,0.5) circle;
\fi
\ifnum#2>1
   \fill[dot hidden] (0.2,0.2) circle;
   \fill[dot hidden] (0.8,0.8) circle;
   \ifnum#2>3
      \fill[dot hidden] (0.8,0.2) circle;
      \fill[dot hidden] (0.2,0.8) circle;
   \fi
   \ifnum#2>5
      \fill[dot hidden] (0.2,0.5) circle;
      \fill[dot hidden] (0.8,0.5) circle;
   \fi
\fi
\end{tikzpicture}
}

