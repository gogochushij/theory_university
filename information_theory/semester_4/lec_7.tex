\lecture{7}{20 May}{\dag} 
\label{proof:thm_4_7_1}
\begin{proof}
Докажем вторую часть \hyperref[thm:4_7_1]{теоремы \ref{thm:4_7_1}}.
Для этого докажем $ I( \pi : x, y) \ge I( \pi : x \mid y) + I(\pi : y  \mid x)$. 

Если оставить только одно слагаемое,  слева останется
$ H( \pi ) - H( \pi \mid x, y) $
, а справа 
$ H( \pi \mid y ) - H( \pi \mid x, y) $, которое точно не больше.

Пусть $  \pi_i$ -- префикс $  \pi$, то есть то, что Алиса и Боб отправили за $ i$-ый раунд ($ i$-ый бит).
\begin{align*}
	I( \pi : x, y) &= \tag{Chain rule} \sum_{i} I( \pi_i : x, y \mid \pi_{< i}) 
\end{align*}
Аналогично попробуем нарезать слагаемые правой части:
\begin{align*}
	I( \pi : x \mid y) & = \sum_{i}^{} I( \pi_i : x \mid y, \pi_{<i}) \\
	I( \pi : y \mid x) & = \sum_{i}^{} I( \pi_i : y \mid x, \pi_{<i}) 
\end{align*}

В ход Боба первое слагаемое <<равно нулю>>\footnote{Это и есть баг, на самом деле это случная величина}, так как $  \pi_i$ определяется $ y $-ом. Аналогично <<второе равно>> нулю, когда ходит Алиса. Поэтому каждый раз неравенство сохраняется. 

Чтобы исправить скрытую фундаментальную ошибку распишем через матожидания
\begin{align*}
	I( \pi : x, y) &= \sum_{i}^{} \E_{m}I( \pi_i : x, y \mid \pi_{< i} = m) \\
	I( \pi : x \mid y) & = \sum_{i}^{} \E_m I( \pi_i : x \mid y, \pi_i = m) \\
	I( \pi : y \mid x) & = \sum_{i}^{} \E_mI( \pi_i : y \mid x, \pi_i = m) 
\end{align*}
Это уже корректное утверждение, так как для каждого конкретного $m$ одно из слагаемых действительно обнуляется.
\end{proof}

\chapter{Колмогоровская сложность}
\section{Определения}

Пусть $ F$ --- вычислимая декодирующая функция.
Определим $ K_F(x) \coloneqq \min \{\lvert p \rvert \mid F(p) = x\}$. Это <<критерий сжимаемости>> функции $ F$.

Будем считать, что $ F \prec G$ ($ F$ не хуже  $ G$ ), если $ \forall x ~ K_F(x) \le K_G(x) + c_{FG}$.

Назовем способ описания (функцию) оптимальным, если она не хуже всех остальных.

\begin{thm}
    Существует оптимальный способ описания.
\end{thm}

\begin{defn}[]
	Колмогоровская сложность для $ x$ --- $ K(x) \coloneqq K_U(x)$, где $ U$ --- оптимальный способ описания.
\end{defn}
\begin{prop}
	~\begin{enumerate}
		\item
	$ K(x) \le \lvert x \rvert +c$, так как можно взять $ K_{id}(x) = \lvert x \rvert $
	I
\item $ K(XX) \le \lvert x \rvert +c$, можно взять описание $ F(p) = pp$ 
\item Пусть $G x$ не более $ p$ единиц, тогда $ K(x) \le H(p) \cdot \lvert x \rvert + c$.

	Можем взять $ F(p) = p\text{-ое слово с не более } p \text{единицами}$.
	\[
		\sum_{i=0}^{n} {i \choose n} \le 2 ^{H(p) n}
	.\] 
	\end{enumerate}
\end{prop}

\begin{thm}
	Пусть $ M$ --- всюду вычислимая функция. Если $ M(x) \le K(x)$ и $ \forall c ~ \exists x \colon M(x) \ge c$, то $ M$ не вычислима.
\end{thm}
\begin{proof}
	Зафиксируем $ c$. Найдем  $ x_c$ --- первое слово, где  $ M(x_c) \ge c$. Так как $ M$ вычислима всюду, определено $ F(c)$. 
	Тогда из $ F(c) = x_c$ следует, что  $ K(x) \le  \log c + c_0$ по определению.
\end{proof}

\begin{cor}
    У почти всех слов колмогоровская сложность равна $ n- const$.
\end{cor}

\section{Применение}
Мы лишаемся вычислимости, поэтому мы лишаемся практических применений, как с энтропией. Но есть математическое применение, так как это оптимальный алгоритм.

Можно доказать, что одноленточная машина Тьюринга, копирующая вход будет работать $ \Omega( \lvert x \rvert ^2)$.

\section{Условная сложность}
\[
	K(x \mid y) = K_{U}\left( x \mid y \right) 
.\] 
Это способ описания, который может еще использовать $ y$ : 
$ K_{F}(x \mid y) \coloneqq \min \{\lvert p \rvert \mid F(p, y) = x\}$. Аналогично $ U$ --- оптимальный способ описания\footnote{Упражнении --- аналогично доказать существования}.

\begin{note}
	$ K(x) - K(x \mid \varnothing) + const$
\end{note}
\[
	K(x, y) \coloneqq K(\langle x, y \rangle) \qquad \langle\cdot  , \cdot \rangle\text{ --- какая-то кодировка пары}
.\] 
\begin{prop}
    ~\begin{enumerate}
		\item $ K(x \mid y) \le  K(x) + \O(1)$
		\item $ K(x, y) \le  K(x) + K(y \mid x)$
    \end{enumerate}
\end{prop}

\begin{thm}[Колмогоров, Левин]
	$ K(x, y) =  K(x) + K(y \mid x)$
\end{thm}
\begin{proof}
	В одну сторону по свойству. Пусть  $ n = K(x, y) $ 
\begin{figure}[ht]
    \centering
    \incfig{kolmo-thm}
    \caption{kolmo thm}
    \label{fig:kolmo-thm}
\end{figure}
Пусть $ S \coloneqq \{(a, b) \mid K(a, b) \le n\}$,
\[
	K(y \mid x) \le  \underbrace{\log \lvert   S_x\rvert}_{m} + \O(\log n)
.\] 
Рассмотрим все $ x$, для которых $ \log \lvert S_x \rvert  \ge  m$. Таких не более $ 2^{n-m + с}$, так как мы знаем, что $ K(x, y) = n$, то внутри множества размером  $ 2^{n}$, есть элемент с большой сложностью, теперь множество $ S$ по размеру не более  $ 2^{n+c}$, но так как в одном сечении не более $ 2^{m}$, получаем  $ 2^{n-m+c}$.

Тогда $ K(x) \le n -m $. Если мы подставим в неравенство выше, то получим то, что хотели.
\end{proof}
Конец.
