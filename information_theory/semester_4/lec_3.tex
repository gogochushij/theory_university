\lecture{3}{15 April}{\dag}
\section{Подсчет углов в графе}
Рассмотрим ориентированный граф.

Назовем треугольником тройку $ (x, y, z)$, если это цикл из трех вершин. Углом назовем тройку  $ (x, y, z)$, если есть ребра  $ xy$ и $xz$, при этом  $ y $ может совпадать  $ z$.

Чего в графе больше: углов или треугольников?
\begin{note}
	Каждое ребро тоже угол, например, $ (x, y, y)$.
\end{note}
\begin{thm}
    Число углов в графе всегда больше числа треугольников.
\end{thm}
\begin{proof}
	Пусть случная величина$ \alpha $ равна случайному треугольнику.

	Так как распределение количества треугольников равномерно,
	\begin{align*}
		\log (\# \Delta)& = H(x, y, z)  = \tag{Chain rule} \\
					  &= H(x) + H(y \mid x) + H(z \mid  y, x) \le  \\
					  & \le  H(x) + H(y \mid x) + H(z \mid y) = \tag{циклический сдвиг в треугольнике} \\
		 & = H(x) + 2 H(Y \mid X)
	\end{align*}
Найдем какое-то распределение на углах, энтропия которого хотя бы $ H(x) + 2H(Y \mid X)$, тогда эта сумма будет не более  $  \log (\# \angle)) $.
	% с картинкой будет лучше!!

	Пусть мы выбрали случайный треугольник $ (x, y, z)$. Оставим $ x$ и выберем для него найдем случайный треугольник с $ x$ и берем из него следующую за $ x$ вершину $ y'$. Повторяем эту операцию еще раз для $ x$ и находим  $ z'$. Тогда $ (x,  y', z')$ --- угол.
	\begin{align*}
		H(x, y', z') & = H(x) + H(y' \mid x) + H(z' \mid x, y') = \tag{Так как $ y'$ и  $ z'$ независимы при выбранном  $ x$}  \\
					 & =H(x) + H(y'\mid x) + H(z' \mid x) = \tag{Выбор аналогичный} \\
					 & = H(x) + 2 H(y' \mid x)
	\end{align*}
	$ H(x)$ здесь совпадает с  $ H(x) $ выше, так как мы выбираем треугольник и вершину аналогично.

	$ y'$ выбирается при фиксированном  $ x$ также, как и выше (выбрали случайный треугольник и в нем вершиной после $ x$ будет  $ y'$).

	Таким образом, мы нашли распределение с такой же энтропией.
\end{proof}

\section{Теория кодирования}
Код --- отображение алфавита $ C \colon \Sigma \to \{0, 1\}^{*}$.

Что хочется требовать?
\begin{enumerate}
    \item Однозначное декодирование. При этом не обязательно у каждой строки $ \{0, 1\}^{*}$ есть слово, но склейки нет.
	\item Префиксный код --- то есть код каждого символа не является префиксом кода другого. Очевидно, из этого следует предыдущий пункт.
\end{enumerate} 

\begin{thm}
    Любой однозначно декодируемый код можно переделать в префиксный с сохранением длин кодовых слов.
\end{thm}
\begin{proof}
	Пусть есть $  c_1, \ldots , c_n$ --- кодовые слова.

	Для префиксного кода $ \sum_{}^{} 2^{-\lvert c_i \rvert } \le 1$, причем, если если выполнено это неравенство, то есть префиксный код.

	Докажем, что для любого декодируемого кода выполнено такое неравенство.

	Построим многочлен для всех слов длины $ L$.

	\[
		p(x, y) = \left( \sum_{i}^{} p_i(x, y) \right) ^{L} = \sum_{j=L}^{} M_j(x, y)
	.\] 
	Здесь $ p_i(x, y)$ --- моном, соответствующий  $ i$-ому символу в алфавите и равный 

	Посчитаем $ p(\frac{1}{2}, \frac{1}{2})$.

	\[
		p(\tfrac{1}{2},\tfrac{1}{2}) = \sum_{j=L}^{} M_{j}(\tfrac{1}{2},\tfrac{1}{2}) \le \sum_{j=L}^{\max_{i} c_i}2^{j} \cdot 2 ^{-j} = \O(L)
	.\] 

	Посчитаем еще раз по второму представлению
	\[
		p(\tfrac{1}{2},\tfrac{1}{2}) = \left( \sum_{i}^{} 2^{-\lvert c_i \rvert } \right) ^{L}
	.\] 
	Если сумма в скобках больше $ 1$, получаем экспоненциальную оценку снизу.
	Следовательно, для больших $ N$ она обгонит линейную. Противоречие.
\end{proof}


\begin{thm}[Шеннон]
Пусть есть множество $ \Sigma $, и с вероятностью $ p_i$ получаем  $ i$-й символ.
   Тогда
   \[
	   \sum_{i}^{} p_i \lvert c_i \rvert  \ge H(p) \qquad  c_i \text{ --- однозначно декодируемы}
   .\] 
\end{thm}
\begin{proof}
	\begin{align*}
		H(p) - \sum_{i}^{} p_i \lvert c_i \rvert &= \sum_{}^{} p_i \log \frac{2^{-\lvert c_i \rvert }}{p_i} \le \tag{Неравенство Йенсена}\\
												 & \le  \log \sum_{}^{} p_i \cdot \frac{2^{-\lvert c_i \rvert }}{p_i} \le \tag{Неравенство Крафта} \\
												 & <= 0
	\end{align*}
\end{proof}


\begin{thm}[Шеннон]
	Существует такой код, что \footnote{Единичка обязательно возникает, так как мы приводим непрерывную энтропию к дискретной величине}
	\[
		\sum_{i}^{} p_i \cdot \lvert c_i \rvert \le H(p) + 1
	.\] 
\end{thm}
\begin{proof}
    Угадаем длины кодов, чтобы выполнялось неравенство ???.
	Пусть $  \lvert c_i \rvert = \lceil \log \frac{1}{p_i} \rceil$,
	\[
		\sum_{}^{} 2^{- \lvert c_i \rvert } = \sum_{}^{} 2^{- \rceil \frac{1}{p_i} \rceil} \le \sum_{}^{} p_i \le 1
	.\] 
\end{proof}

\section{Код Шенона-Фано}
Отсортируем вероятности по убыванию $  p_1 \ge p_2 \ge  \ldots \ge p_n$. Затем уложим их в отрезок $ [0, 1]$.

Разделим отрезок пополам и скажем, что слева кодовые слова начинается с  $ 0$, справа с  $ 1$, а центральный  $ p_i$ будет начинаться с нуля, если это $ p_1$, с единицы, если $ p_n$, и, наконец, иначе выдираем любое значение.

Далее рекурсивно запускаемся на группе нулей и на группе единиц.

Когда остался один кусок, останавливаемся.

\begin{thm}
	\[
		\sum_{0}^{n} p_i \cdot \lvert c_i \rvert  \le H(p) + \O(1), \quad n \to \infty, ~\O(1) \approx 3 \text{ или } 5 
	.\] 
\end{thm}
\begin{proof*}
    Упражнение со зведочкой.
\end{proof*}


\section{Код Хаффмана}
Опять отсортируем $  p_1 \ge  p_2 \ge  \ldots \ge p_n$. Возьмем $ p_{n-1}$ и $ p_n$. Заменим их на один символ с вероятностью  $ p_n + p_n-1$, теперь по индукции строим код для  $ n-1$ символа.

Теперь если объединенному символу соответствовал код  $  \overline{c}$, то для $ p_{n-1}$ задаем код  $ \overline{c 0}$, а для $ p_n$ код  $ \overline{c 1}$
 
Проверим, что $ \sum_{i=1}^{n} p_i \lvert c_i \rvert \le H(p) + 1$, причем $  \forall c'_i \colon \sum_{i=1}^{0} p_i \lvert c_i \rvert  \le \sum_{i=1}^{} p_i \lvert c_i \rvert $

Достаточно доказать второе, а потом сравнить с кодом Шеннона и получить нужное неравенство.

Рассмотрим набор $  c_1', \ldots c_n'$.  Возьмем два минимальных $ c_{n-1}'$ и $ c_n'$. Заметим, что можно поменять их с символами максимальной длины  $ c_i'$ и $ c_j'$, при этом длина кода не увеличится.

Изучим коды $ c_{n-1}'$ и $ c_n'$. Пусть они не имеют вид $ \overline{v 0}$ и $ \overline{v 1}$.
\begin{itemize}
	\item Пусть $ \lvert c_{n-1}' \rvert  \le  \lvert c_n' \rvert $. Посмотрим на $ c_{n-1}'$: не умаляя общности он будет заканчиваться на $ 0$ ( $ \overline{s 0}$ ). Заменим $ c_n'$ на $ s 1$. Если вдруг кто-то уже имел такой код, это и есть  $ c_n'$, так как имеет максимальную длину.

	% Написать подробнее!!!
\end{itemize}

\section{Арифметическое кодирование}
Уложим вероятности аналогично не отрезок, при этом не обязательно в порядке убывания.

Назовем \selectedFont{стандартным} интервал $ \bigl[\overline{0 v 0}, \overline{0 v 1}\bigr)$. Найдем максимальный стандартный интервал в отрезке $ p_i$. Тогда $ v$ будет кодом  $ p_i$.

\begin{proof*}
	Если рассмотреть отрезок $ [a, b]$, есть стандартный интервал длиной  $ \frac{b-a}{8}$. Упражнение
\end{proof*}

